% Created 2022-04-02 Sat 14:39
% Intended LaTeX compiler: pdflatex
\documentclass[a4paper, 11pt, twocolumn, twoside]{article}
\usepackage[utf8]{inputenc}
\usepackage[T1]{fontenc}
\usepackage{graphicx}
\usepackage{grffile}
\usepackage{longtable}
\usepackage{wrapfig}
\usepackage{rotating}
\usepackage[normalem]{ulem}
\usepackage{amsmath}
\usepackage{textcomp}
\usepackage{amssymb}
\usepackage{capt-of}
\usepackage{hyperref}
\usepackage{baskervillef}
\usepackage{geometry}\geometry{ a4paper, total={170mm,257mm}, left=20mm, top=20mm,}
\usepackage{hyperref}\hypersetup{pdfauthor={Olivier Rey}, pdftitle={Risus - Règles Compressées - Version Française}, pdfkeywords={jdr, risus}, pdfsubject={jeu de rôles}, pdfcreator={Emacs 26.1 (Org mode 9.1.9)}, pdflang={Frenchb}, colorlinks=true, linkcolor={blue}, urlcolor={blue}}
\usepackage[french, frenchb]{babel}
\usepackage{titlesec}\titlelabel{\thetitle. \quad}
\usepackage[table,svgnames]{xcolor}\rowcolors{1}{Gainsboro}{WhiteSmoke}
\usepackage{etoolbox}\AtBeginEnvironment{longtable}{\small}
\author{Olivier Rey}
\date{2022-04-02}
\title{Risus - Règles compressées}
\hypersetup{
 pdfauthor={Olivier Rey},
 pdftitle={Risus - Règles compressées},
 pdfkeywords={},
 pdfsubject={},
 pdfcreator={Emacs 26.1 (Org mode 9.1.9)}, 
 pdflang={Frenchb}}
\begin{document}

\maketitle
\tableofcontents

\newpage

\begin{center}
\includegraphics[width=4cm]{logo-risus.png}
\end{center}

\section{Introduction}
\label{sec:org85e33ac}

\begin{longtable}{ll}
Version originale & \href{https://www.drivethrurpg.com/product/170294/Risus-The-Anything-RPG}{Risus the RPG}\\
Version française & \href{risus-fr.pdf}{Risus, traduction de Tristan Lhomme}]\\
Site américain & \href{http://www.risusiverse.com/}{risuiverse}\\
Compression des règles & Olivier Rey\\
Copyleft & 2020-2022\\
\end{longtable}


\section{V1 du flowchart Risus : création de perso + système de jeu}
\label{sec:org051a47f}

\href{thumbnail-risus-flowchart.png}{Thumbnail du diagramme}

Publié sur \href{https://rouboudou.itch.io/risus-flowchart}{itch.io}.

Versions :

\begin{itemize}
\item \href{risus-flowchart.pdf}{En PDF},
\item \href{risus-flowchart.png}{En PNG taille normale}
\item \href{risus-flowchart-big.png}{En PNG taille maxi}
\end{itemize}

\section{Création du personnage}
\label{sec:orga17e3f3}

\subsection{Les clichés}
\label{sec:org7a50bdc}

Chaque personnage a un crédit de 10D6 à répartir sur des Clichés (entre 3 et 10) librement choisi.

Pour information :

\begin{longtable}{lc}
Niveau & Nombre de dés\\
Newbie & 1D\\
Pro & 3D\\
Maître & 6D\\
\end{longtable}

Les clichés n'ont pas tous le même coût :

\begin{longtable}{l p{1.5cm} c p{08cm] p{3cm} l}
Type & Coût du cliché & Gonflette & Double gonflette & Effet gonflette & Notation\\
Normal & 1D investi = 1D de cliché & Accord MJ & Non & +1D en combat, -1D sur le cliché au round suivant & Entre parenthèses\\
Magique & 2D investis = 1D de cliché & Accord MJ & Oui & +2nD en combat, -nD sur le cliché au round suivant & Entre crochets\\
\end{longtable}

Note : Les clichés magiques peuvent être "doublement gonflés" en combat (voir partie combat).

\subsection{Matériel}
\label{sec:orgc2fd8cb}

Inclus quand dans cela reste dans le raisonnable.

\subsection{Coup de bol (option)}
\label{sec:org426afeb}

1D permet d'acheter 3 coups de bol :
\begin{itemize}
\item 1 coup de bol = +1D sur une action
\item 1D = 3 coups de bols
\end{itemize}

\subsection{Point faible (nommé "Amorce")}
\label{sec:org5294fd2}

+1D de création si validation du point faible par le MJ.

\subsection{Historique}
\label{sec:org637614c}

S'il est bien fait, le MJ peut donner +1D.

\section{Action simple : jet contre un facteur de difficulté}
\label{sec:org942ac81}

Ci-dessous, la table de difficultés.

\begin{longtable}{cl}
Seuil à dépasser & Difficulté de l'action\\
5 & Facile\\
10 & Défi même pour un pro\\
15 & Défi héroïque\\
20 & Difficulté presque surhumaine\\
30 & Difficulté surhumaine\\
\end{longtable}


\section{Combat : basé sur le système du duel}
\label{sec:org1474ce0}

\subsection{Types de combats}
\label{sec:org4062b1c}

Exemples de cas d'utilisation :

\begin{longtable}{l}
Type de combat\\
Débat\\
Courses de chevaux\\
Duel aérien\\
Duels astraux / psychiques\\
Duels magiques\\
Duels de Banjos\\
Séduction\\
Tribunal\\
Combat physique\\
Etc.\\
\end{longtable}

Généralement, l'agresseur détermine le type de combat. Du type de combat dépend le type de cliché utilisé.

\subsection{Round de combat}
\label{sec:orgfb0b9d7}

\begin{longtable}{c c p{3cm} p{3cm}}
Etape & Obligatoire & Description & Conséquence\\
1 & Oui & Choisir le cliché & Le MJ détermine si le cliché est adapté ou pas (1)(2)\\
2 & Non & Gonflette & +nD au jet / -nD sur le cliché à la fin du round\\
 &  & Double gonflette (magie) & +2nD au jet / -nD sur le cliché à la fin du round\\
3 & Oui & Lancer les dés & \\
4 & Oui & Perdant du round & Cliché adapté : -1D\\
 &  &  & Cliché inadapté : -3D\\
5 & Oui & Si un combattant a un cliché à 0D, il a perdu & Le vainqueur fait ce qu'il veut du perdant\\
\end{longtable}

(1) Notes sur les clichés inadaptés :
\begin{itemize}
\item Le roleplay permet de l'utiliser (donc même tiré par les cheveux, ce doit être utilisable) ;
\item -3D si le cliché inadapté gagne.
\end{itemize}

(2) Porte de sortie : 2D par défaut en cas d'absence totale de cliché adapté.

\subsection{Récupération}
\label{sec:org5d99df3}

On récupère les dés avec de la guérison (contextuelle au type de duel).

\subsection{Conflit à action unique (actions très rapides)}
\label{sec:orgda54926}

Un seul jet sur le Cliché approprié.

\section{Groupes}
\label{sec:org451c24b}

\subsection{Groupe de PNJ}
\label{sec:orgbcada5f}

Le groupe de PNJ se comporte comme un PNJ mais avec plus de dés.

\subsection{Groupe de PJ}
\label{sec:org5b87b7c}

Groupe de PJ :

\begin{itemize}
\item Le Chef de Groupe est celui dont le le Cliché s'applique et qui a le plus de dés.
\item Les jets des autres ne comptent en plus du Chef de groupe que s'ils font 6.
\item Les clichés inadaptés ne triplent pas les pertes de dés.
\item Si tous les clichés sont inadaptés et perdent, discuter avec le MJ.
\item Si le Groupe perd, le Cliché d'un seul membre est diminué (nommé par le Chef de Groupe). Si quelqu'un veut bien prendre les dommages à sa place, ce quelqu'un perd 2D sur son Chiché mais le Chef de Groupe a le droit de lancer \emph{deux fois plus de dés} pour la prochaine attaque (bonus de vengeance).
\end{itemize}

Si quelqu'un sort du Groupe, tous les membres du Groupe perdent un dé et celui qui sort du Groupe, passe à 0D. Si le Chef qui part, consulter les règles.

\section{Expérience}
\label{sec:orge8276f1}

A la fin, jet de Cliché: si tous les dés sont pairs, +1D au cliché (6D max par cliché). Ou nouveau Cliché à 1D.

Ce jet peut être fait en cas d'action exceptionnelle durant le jeu.


\vfill

\begin{center}
\includegraphics[width=3cm]{logo-orey.png}
\end{center}
\end{document}
