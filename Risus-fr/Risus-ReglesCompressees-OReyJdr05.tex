\documentclass{article}

\usepackage[utf8]{inputenc}
\usepackage[french]{babel}

\usepackage[a3paper,margin=1in,landscape]{geometry}

\usepackage[table]{xcolor}
\definecolor{lightgray}{gray}{0.85}
\definecolor{verylightgray}{gray}{0.95}
\let\oldtabular\tabular
\let\endoldtabular\endtabular
\renewenvironment{tabular}{\rowcolors{1}{lightgray}{verylightgray}\oldtabular}{\endoldtabular}

% Pour le bas de page
\newcommand{\mysmallgray}[1]{\scriptsize\color{gray}#1}

\usepackage{comment}

\usepackage{multicol}
\setlength{\columnsep}{0.5cm}
\usepackage{wrapfig}

\usepackage{graphicx}

\usepackage{hyperref}
\hypersetup{
colorlinks=true,
linkcolor=blue,
urlcolor=blue,
}

\usepackage{fancyhdr}
\pagestyle{fancy}
\fancyhead[C]{{\color{violet}\textbf{{\Huge R}{\LARGE ISUS - }{\Huge R}{\LARGE ÈGLES }{\Huge C}{\LARGE OMPRESSÉES}}}}

\fancyfoot[L]{\mysmallgray{Version 1.0}}
\fancyfoot[C]{\mysmallgray{\today}}
\fancyfoot[R]{\mysmallgray{Copyleft \href{https://github.com/orey/jdr}{Olivier Rey}}}
\renewcommand{\headrulewidth}{0.4pt}
\renewcommand{\footrulewidth}{0.4pt}

% Enlève l'indentation pour tout le documnt (équivalent de \noindent sur toutes les lignes)
\setlength\parindent{0pt}

% Mes macros

\newcommand{\mysection}[1]{
\vspace{0.2cm}
\noindent{\color{violet}\large\textbf{#1}}
}

\newcommand{\mysubsection}[1]{
\vspace{0.1cm}
\noindent{\textit{\textbf{#1}}}
}


%=======================================DOC
\begin{document}

\begin{multicols*}{3}

\begin{center}
\includegraphics[scale=0.30]{logo-risus}
\end{center}

%\mysection{Introduction}

%\begin{wraptable}{l}{0.43\linewidth} % Pour wrapper le paragraphe à côté
\begin{center}
\begin{tabular}{ll}
Version originale & \href{https://www.drivethrurpg.com/product/170294/Risus-The-Anything-RPG}{Risus the RPG} (c) John Ross \\
Version française & \href{https://github.com/orey/jdr/blob/master/Risus-fr/risus-fr.pdf}{Risus, traduction de Tristan Lhomme} \\
Site américain         & \href{http://www.risusiverse.com/}{risuiverse} \\
Compression            & Olivier Rey \\
Copyleft               & 2020-2022  \\
\end{tabular}
\end{center}
%\end{wraptable}

\vspace{0.1cm}

%======= mysection
% TODO Flowchart
\mysection{V1 du flowchart Risus}

N'hésitez pas à vous procurer le flowchart Risus sur \href{https://rouboudou.itch.io/risus-flowchart}{itch.io}.

%======= mysection
% TODO Création du personnage
\mysection{Création du personnage}

%--- mysubsection
\mysubsection{Les clichés}

Chaque personnage a un crédit de 10D6 à répartir sur des Clichés (entre 3 et 10) librement choisi.

La table ci-dessous montre le nombre de dés par niveau :

\begin{center}
\begin{tabular}{lc}
\textbf{Niveau du personnage} & \textbf{Nombre de dés} \\
Débutant &  1D \\
Professionnel & 3D \\
Expert, maître & 6D \\
\end{tabular}

\end{center}
Attention, tous les clichés n'ont pas tous le même coût :

\begin{center}
\begin{tabular}{llc}
\textbf{Type}    & \textbf{Coût du cliché} & \textbf{Gonflette ?}  \\
Normal  & 1D investi = 1D de cliché  & Simple (1) \\
Magique & 2D investis = 1D de cliché & Double (2) \\
\end{tabular}
\end{center}

(1) Gonflette (simple) :
\begin{itemize}
\item Accord du MJ,
\item +nD pour un round de combat, -nD sur le cliché à partir du round suivant,
\item Notation de la compétence entre parenthèses.
\end{itemize}

(2) Double gonflette (pour les clichés magiques) :
\begin{itemize}
\item Accord du MJ,
\item +2nD pour un round de combat, -nD sur le cliché à partir du round suivant,
\item Notation de la compétence entre crochets.
\end{itemize}

%--- mysubsection
\mysubsection{Matériel}

Inclus quand dans cela reste dans le raisonnable.

%--- mysubsection
\mysubsection{Coup de bol (option)}

1D permet d'acheter 3 coups de bol :
\begin{itemize}
\item 1 coup de bol = +1D sur une action
\item 1D = 3 coups de bols
\end{itemize}

%--- mysubsection
\mysubsection{Point faible}

+1D de création si validation du point faible par le MJ.

%--- mysubsection
\mysubsection{Historique/background}

S'il est bien fait, le MJ peut donner +1D.

%======= mysection
% TODO Action simple
\mysection{Action simple : jet contre un facteur de difficulté}

\begin{center}
\begin{tabular}{cl}
\textbf{Seuil à dépasser} & \textbf{Difficulté de l'action}        \\
               5 & Facile                       \\
              10 & Défi même pour un pro         \\
              15 & Défi héroïque                 \\
              20 & Difficulté presque surhumaine \\
              30 & Difficulté surhumaine         \\
\end{tabular}
\end{center}

%======= mysection
% TODO Combat
\mysection{Combat : basé sur le système du duel}

\mysubsection{Types de combats}

Exemples de cas d'utilisation :

\begin{center}
\begin{tabular}{cc}
\textbf{Type de combat} & \textbf{Type de combat} \\
Débat & Courses de chevaux  \\
Duel aérien & Duels astral \\
Duel psychique & Duel de banjos \\
Séduction & Tribunal \\
Combat physique & Etc. \\
\end{tabular}

\end{center}

Généralement, l'agresseur détermine le type de combat. Du type de combat dépend le type de cliché utilisé.

\mysubsection{Round de combat}

\begin{center}
\begin{tabular}{cp{4.5cm}p{5.5cm}}
\textbf{\#} & \textbf{Description} & \textbf{Conséquence} \\
1 & Choisir le cliché & Le MJ détermine si le cliché est adapté ou pas (1)(2) \\
2 & Option : gonflette & +nD au jet / -nD sur le cliché à la fin du round  \\
  & Option : double gonflette (magie) & +2nD au jet / -nD sur le cliché à la fin du round \\
3 & Lancer les dés & \\
4 & Perdant du round  & Cliché adapté : -1D \\
  &                   & Cliché inadapté : -3D \\
5 & Si un combattant a un cliché à 0D, il a perdu & Le vainqueur fait ce qu'il veut du perdant \\
\end{tabular}

\end{center}
(1) Notes sur les clichés inadaptés :
\begin{itemize}
\item Le roleplay permet de l'utiliser (donc même tiré par les cheveux, ce doit être utilisable) ;
\item -3D si le cliché inadapté gagne.
\end{itemize}

(2) Porte de sortie : 2D par défaut en cas d'absence totale de cliché adapté.

\mysubsection{Récupération}

On récupère les dés avec de la guérison (contextuelle au type de duel).

%======= mysection
% TODO Conflits à action unique
\mysection{Conflits à Action Unique (CAU) (action très rapide)}

Le CAU fonctionne comme un combat normal mais avec un seul jet de dés.

%======= mysection
% TODO Groupes
\mysection{Groupes}

\mysubsection{Groupe de PJ}

\begin{center}
\begin{tabular}{cp{2cm}p{8	cm}}
\textbf{\#} & \textbf{Description} & \textbf{Conséquence} \\
1 & Choisir le Chef De Groupe (CDG) & Le CDG est celui dont le Cliché s'applique et qui a le plus de dés \\
  &  & En cas d'égalité, les joueurs désignent leur chef \\
2 & Déterminés les Clichés adaptés & Généralement, celui chef de groupe et un mélange de Clichés adaptés et inadaptés (si le MJ est d'accord) \\
3 & Faire le jet & Le Cliché du CDG compte \\
  &  & Les Clichés adaptés/inadaptés des autres joueurs comptes uniquement s'ils font des 6 \\
4 & Round de combat perdu & Un des membres doit se porter volontaire pour prendre les dommages (2D si au moins un des Clichés est adapté, 6D sinon) \\
  &  & Si un PJ s'est désigné, le groupe a droit à un bonus de vengeance (3). Dans le cas contraire, le CDG désigne celui qui prend les dommages (pas de bonus) \\
\end{tabular}
\end{center}

(3) Bonus de vengeance : le Groupe a le droit de lancer deux fois plus de dés lors du round suivant pour venger le membre du Groupe qui a pris les dommages.

\textit{Issue du combat:}
\begin{itemize}
\item Lorsqu'un membre du groupe passe à zéro durant le combat, on attend en général la fin du combat pour se préoccuper de son sort (savoir si le groupe est vainqueur ou pas).
\item Débandade du Groupe : tous les membres du Groupe perdent 1D pour le prochain round .
\item Un membre quitte le Groupe : il se retrouve à 0D sur son Cliché et est à la merci du vainqueur.
\item Si le CDP quitte le Groupe, c'est la débandade.
\item Si un autre Groupe se reforme alors que le CDP du Groupe précédent a pris des dommages et est passé à OD, alors le nouveau groupe a droit à un bonus de vengeance (3).
\end{itemize}

\mysubsection{Groupe de PNJ}

Le groupe de PNJ se comporte comme un PNJ mais avec plus de dés.

%======= mysection
% TODO Participation impossible
\mysection{Participation impossible des PJ à un combat}

Lorsqu'aucun Cliché ne fonctionne pour les PJ, le MJ peut attribuer aux deux adversaires 2D temporaires (donc 2D pour les PJ et +2D pour les PNJ).

%======= mysection
% TODO Expérience
\mysection{Expérience}

A la fin du jeu, jet de Cliché sur les Clichés utilisés durant le jeu: si tous les dés sont pairs, +1D au cliché (6D max par Cliché).

Avec accord du MJ, il est possible d'utiliser le dé gagné pour créer un nouveau Cliché à 1D. Cela peut même être fait en cas d'action exceptionnelle durant le jeu.

%======= mysection
% TODO licence
\mysection{Licence}

Risus : The Anything RPG est une marque déposée par S. John Ross pour son jeu de rôle de tout (\textit{The Anything RPG}). Voir aussi la traduction française de Tristan Lhomme. ©1993-2013 par S. John Ross, tous droits réservés.

%======= mysection
% TODO licence
\mysection{Appendix: Risus pour \href{https://www.messagers-galactiques.com/joomla/}{Méga}}

Tous les joueurs démarrent avec 3D dans un Cliché Méga. Voici quelques exemples de variantes :




\begin{center}
\includegraphics[scale=0.7]{logo-orey}
\end{center}

\end{multicols*}



\end{document}
