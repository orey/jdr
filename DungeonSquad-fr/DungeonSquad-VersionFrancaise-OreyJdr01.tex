% Created 2021-10-03 Sun 23:27
% Intended LaTeX compiler: pdflatex
\documentclass[a4paper, 11pt, twoside]{article}
\usepackage[utf8]{inputenc}
\usepackage[T1]{fontenc}
\usepackage{graphicx}
\usepackage{grffile}
\usepackage{longtable}
\usepackage{wrapfig}
\usepackage{rotating}
\usepackage[normalem]{ulem}
\usepackage{amsmath}
\usepackage{textcomp}
\usepackage{amssymb}
\usepackage{capt-of}
\usepackage{hyperref}
\usepackage{geometry}\geometry{ a4paper, total={170mm,257mm}, left=20mm, top=20mm,}
\usepackage{hyperref}\hypersetup{pdfauthor={Olivier Rey}, pdftitle={Dungeon Squad! - Version Française}, pdfkeywords={jdr, dungeonsquad}, pdfsubject={jeu de rôles}, pdfcreator={Emacs 26.1 (Org mode 9.1.9)}, pdflang={Frenchb}, colorlinks=true, linkcolor={blue}, urlcolor={blue}}
\usepackage[french, frenchb]{babel}
\usepackage{titlesec}\titlelabel{\thetitle. \quad}
\author{Olivier Rey}
\date{2021-10-03}
\title{Dungeon Squad! - Version Française}
\hypersetup{
 pdfauthor={Olivier Rey},
 pdftitle={Dungeon Squad! - Version Française},
 pdfkeywords={},
 pdfsubject={},
 pdfcreator={Emacs 26.1 (Org mode 9.1.9)}, 
 pdflang={Frenchb}}
\begin{document}

\maketitle
\tableofcontents

\newpage

\begin{center}
\includegraphics[width=.9\linewidth]{logo.png}
\end{center}

\section{Dungeon Squad! - Version Française}
\label{sec:org4a12f73}

\subsection{Introduction}
\label{sec:org710cee6}

Dungeon Squad est un jeux de rôles conçu pour les jeunes joueurs qui ont des plages d'attention réduites et qui demandent de l'action et du fun. Il faut jeter beaucoup de dés et il est possible de faire des choses amusantes comme du shopping ou des calculs. Les personnages peuvent être créés en 30 secondes.

\begin{longtable}{|l|l|}
\hline
Concepteur & (C) Jason Morningstar\\
Version traduite & 2005\\
PDF original & \href{https://github.com/orey/jdr/blob/master/DungeonSquad-fr/dungeon\_squad.pdf}{dungeonsquad.pdf}\\
\hline
\endfirsthead
\multicolumn{2}{l}{Suite de la page précédente} \\

PDF original & \href{https://github.com/orey/jdr/blob/master/DungeonSquad-fr/dungeon\_squad.pdf}{dungeonsquad.pdf} \\

\hline
\endhead
\hline\multicolumn{2}{r}{Suite page suivante} \\
\endfoot
\endlastfoot
\hline
Version & 1.0\\
\hline
(C) traduction et adaptation & O. Rey 2021\\
 & Le jeu a été remanié dans sa présentation pour être plus clair.\\
 & Les règles ajoutées ont été signalées.\\
Octobre 2021 & Corrections diverses\\
 & Intégration du générateur de rencontres\\
 & Mise en place des tags pour génération Latex/PDF\\
\hline
\end{longtable}

\subsection{Ce dont vous avez besoin pour jouer}
\label{sec:orgf3aa2eb}

Chaque joueur doit avoir accès à :
\begin{itemize}
\item Un crayon et du papier,
\item Un ensemble de dés polyhédriques par personne : 1D4, 1D6, 1D8, 1D10 et 1D12 ;
\item Des choses à picorer et des boissons.
\end{itemize}

\subsection{Note}
\label{sec:orgacb6df8}

Un certain nombre de ressources sont disponibles à la fin de ce document :
\begin{itemize}
\item Des liens vers des feuilles de personnage et un écran ;
\item Un générateur de rencontres est proposé à la fin de ce jeu. En l'utilisant avec le \href{https://github.com/orey/jdr/tree/master/G\%25C3\%25A9n\%25C3\%25A9rateurLabyrinthe}{générateur de labyrinthe}, il est possible créer un genre de jeu de plateau dynamique et aléatoire.
\end{itemize}

\section{Création des personnages}
\label{sec:org3774562}
\subsection{Caractéristiques}
\label{sec:org884be30}

Chaque personnage possèdes trois caractéristiques : \textbf{Magicien}, \textbf{Guerrier} et \textbf{Voleur} (respectivement \textbf{Magicienne}, \textbf{Guerrière} et \textbf{Voleuse}). Chaque joueur possède 3 dés, un D4, un D8 et un D12, chaque dé devant être assigné à une des trois caractéristiques. 

\emph{Exemple 1 : Luna de Balor est une magiciene voleuse. Sa joueuse a choisi :}

\begin{longtable}{l|c}
\hline
Magicienne & D12\\
Voleuse & D8\\
Guerrière & D4\\
\end{longtable}

\emph{Espérons que j'aurai quelqu'un près de moi pour me protéger durant les combats !}

\emph{Exemple 2 : Grognard le Turbulent est un guerrier. Son joueur a choisi la configuration suivante :}

\begin{longtable}{l|c}
\hline
Guerrier & D12\\
Voleur & D8\\
Magicien & D4\\
\end{longtable}


\subsection{Compétences}
\label{sec:org7f3aec1}
\subsubsection{Compétences à la création du personnage}
\label{sec:org934ea53}

Chaque personnage possède un D6 et un D10 pour décrire des compétences.

Les compétences peuvent être :
\begin{itemize}
\item Une arme ;
\item Une armure ;
\item Un sort.
\end{itemize}

Au début du jeu, les personnages ont seulement \textbf{2} compétences. Ils affectent à chacune d'elles un des deux dés (D6 et D10).

\emph{Exemple : la joueuse de Luna de Balor a choisi les compétences suivantes :}

\begin{longtable}{l|l|c}
\hline
Sort & Eclair & D10\\
Sort & Soins & D6\\
\end{longtable}

\emph{Exemple : Grognard le Turbulent a les compétences suivantes :}

\begin{longtable}{l|l|c}
\hline
Arme & Epée à deux mains & D10\\
Armure & Cotte de mailles & D6\\
\end{longtable}

\emph{Une autre option pour le guerrier ayant le D4 assigné à la caractéristique \textbf{Magicien} de prendre le sort \textbf{Guérison}, en ne le lançant qu'entre les combats.}

\begin{longtable}{l|l|c}
\hline
Arme & Epée à deux mains & D10\\
Sort & Soins & D6\\
\end{longtable}

\subsubsection{Limitation à 4 du nombre de compétences}
\label{sec:orge8d1c6d}

Dans le cours du jeu, les joueurs pourront acquérir d'autres compétences en trouvant, par exemple, des objets magiques ou des armes. Pour autant, ils ne pourront \textbf{jamais avoir plus de 4} compétences (en comptant les armes, les armures et les sorts).

\emph{Par exemple, si le personnage trouve un trésor avec quelque chose d'intéressant à l'intérieur (comme un parchemin de \textbf{Boule de feu} par exemple), il l'inscrit sur sa feuille en plus des deux autres choses qu'il a choisies à la création de son personnage. Le personnage peut choisir de prendre ou de laisser des choses, mais il ne peut pas en avoir plus de 4 avec lui.}

\subsection{Points de vie}
\label{sec:orga0c9b3a}

Tous les personnages ont \textbf{15 points de vie} (PV). Les dommages réduisent directement les PV (après prise en compte de l'armure).

\subsection{Pièces d'or}
\label{sec:org044e055}

Chaque personnage démarre dans la vie avec \textbf{30 PO} (pièce d'or). Il peut s'acheter un équipement avec cette somme (voir la partie équipement).

\section{La mécanique du jeu}
\label{sec:orgf18f17b}
\subsection{Les seuils de difficulté}
\label{sec:org59458ee}
Tous les personnages peuvent se battre, lancer des sorts et fureter. La difficulté de l'action est déterminée par un seuil qu'il faut dépasser avec le dé assigné à l'action.

\begin{longtable}{c|c}
Difficulté & Seuil\\
\hline
\endfirsthead
\multicolumn{2}{l}{Suite de la page précédente} \\
\hline

Difficulté & Seuil \\

\hline
\endhead
\hline\multicolumn{2}{r}{Suite page suivante} \\
\endfoot
\endlastfoot
\hline
Facile & 2\\
Moyen & 4\\
Difficile & 6\\
Très difficile & 8\\
\end{longtable}

\subsection{Initiative (règle optionnelle)}
\label{sec:orgd72b511}

Les règles originales ne proposent pas de règles concernant l'initiative. Or, savoir qui attaque en premier est important dans un combat.

Nous proposons la règle optionnelle suivante : Lancez 1D6 par joueur et ajoutez le modificateur suivant, dépendant du dé de \textbf{voleur} :

\begin{longtable}{c|c}
Dé de voleur pour les PJ & Modificateur\\
\hline
\endfirsthead
\multicolumn{2}{l}{Suite de la page précédente} \\
\hline

Dé de voleur pour les PJ & Modificateur \\

\hline
\endhead
\hline\multicolumn{2}{r}{Suite page suivante} \\
\endfoot
\endlastfoot
\hline
D4 & -1\\
D8 & +1\\
D12 & +2\\
\end{longtable}

Pour les monstres ajoutez au D6 le modificateur présent dans la table de monstres.

Classez alors l'ordre des attaques et résolvez le tour.

Cette règles optionnelle marche aussi avec les magiciens usant d'un sort d'attaque (comme "éclair").

\subsection{Combat}
\label{sec:org459fcdc}
\subsubsection{Toucher}
\label{sec:org4570930}

Les monstres sont caractérisés par un niveau : faible, moyen ou fort.

Dans un combat, lancez votre dé de \textbf{Guerrier} et consultez la table ci-dessous :

\begin{longtable}{c|c|c|c}
Catégorie de monstre & Dé & Seuil à dépasser & Restriction armes de jet\\
\hline
\endfirsthead
\multicolumn{4}{l}{Suite de la page précédente} \\
\hline

Catégorie de monstre & Dé & Seuil à dépasser & Restriction armes de jet \\

\hline
\endhead
\hline\multicolumn{4}{r}{Suite page suivante} \\
\endfoot
\endlastfoot
\hline
Faible & Guerrier & 2 & Toucher pour nombre pair seulement\\
Moyen & Guerrier & 4 & Toucher pour nombre pair seulement\\
Fort & Guerrier & 6 & Toucher pour nombre pair seulement\\
Au secours ! & Guerrier & 8 & Toucher pour nombre pair seulement\\
\end{longtable}

Explication du tableau ci-dessus : dans le cas d'un combat au corps à corps, il n'y a pas de restriction. Dans le cas de l'utilisation d'une arme de jet, il faut à la fois faire plus que le nombre à dépasser mais encore tirer un nombre pair pour toucher. 

Bien entendu, si vous avez assigné le D4 au \textbf{Guerrier}, le combat risque de s'avérer difficile, voire impossible dans certains cas (cas des monstres forts).

\newpage

\subsubsection{Dommages}
\label{sec:org4710de9}

POur faire des dommages, il faut avoir choisi une arme comme compétence (donc soit lui avoir affecté un D6, soit un D10).

\begin{longtable}{l|c}
Type d'arme & Dommages\\
\hline
\endfirsthead
\multicolumn{2}{l}{Suite de la page précédente} \\
\hline

Type d'arme & Dommages \\

\hline
\endhead
\hline\multicolumn{2}{r}{Suite page suivante} \\
\endfoot
\endlastfoot
\hline
Poings & 1D4\\
Dague & 1D4\\
Epée & 1D6\\
Epée magique & 1D8\\
Hache & 1D8\\
Arc & 1D6\\
Arbalète & 1D8\\
Lance & 1D8\\
\end{longtable}

\subsubsection{Armures}
\label{sec:org73c6c50}

Les armures réduisent les dommages encaissés de la valeur de leur dé assigné.

\emph{Par exemple, dans un combat, si vous prenez 7 points de dommages, que vous avez un armure D6, et que vous faites 4 à votre jet, vous prendrez seulement 3 points de dommages.}

D'autres armures peuvent être trouvées dans les trésors avec des dés pouvant aller du D4 au D12.

Pour utiliser une armure, il faut avoir affecté au préalable un dé de compétence à une armure.

\begin{longtable}{l|c}
Type d'armure & Protection\\
\hline
\endfirsthead
\multicolumn{2}{l}{Suite de la page précédente} \\
\hline

Type d'armure & Protection \\

\hline
\endhead
\hline\multicolumn{2}{r}{Suite page suivante} \\
\endfoot
\endlastfoot
\hline
Armure de cuir & 1D4\\
Armure de cuir renforcé & 1D6\\
Cotte de mailles & 1D8\\
Armure à plaques & 1D10\\
\end{longtable}

\subsection{Magie}
\label{sec:orgfbd939f}
\subsubsection{Lancer un sort}
\label{sec:orgdc8e505}

Pour lancer un sort, jetez votre dé de \textbf{Magicien} :

\begin{longtable}{l|c}
Situation du magicien & Seuil à dépasser\\
\hline
\endfirsthead
\multicolumn{2}{l}{Suite de la page précédente} \\
\hline

Situation du magicien & Seuil à dépasser \\

\hline
\endhead
\hline\multicolumn{2}{r}{Suite page suivante} \\
\endfoot
\endlastfoot
\hline
Dans un combat & 6\\
Hors d'un combat & 2\\
\end{longtable}

\subsubsection{Liste de sorts}
\label{sec:org33c9bca}

6 sorts sont à disposition. Chaque joueur peut choisir d'assigner un dé de choses (D6 ou D10) à un sort, voire les deux dés de compétences à deux sorts différents.

\newpage

\begin{longtable}{l|p{10cm}|c}
Sort & Description & Fréquence\\
\hline
\endfirsthead
\multicolumn{3}{l}{Suite de la page précédente} \\
\hline

Sort & Description & Fréquence \\

\hline
\endhead
\hline\multicolumn{3}{r}{Suite page suivante} \\
\endfoot
\endlastfoot
\hline
Eblouissement & Désoriente un ennemi de la taille d'un humain par 2 points de jet, 4 pour les grosses créatures et 1 pour les petites. Les victimes ne peuvent pas agir pendant un tour. & 1 fois par combat\\
Boule de feu & Les dommages sur la cible sont le triple du dé assigné. Tous ceux qui sont proches de la cible prennent le dé de dommages (sans multiplicateur). & 1 fois par aventure\\
Soins & Fournit le dés du sort en PV à la personne guérie. Ne s'applique qu'à une seule personne. & 1 fois par combat\\
Eclair & Les dommages sont ceux du dé assigné. Le magicien peut diviser les dommages sur plusieurs cibles. & Chaque tour\\
Chance & Permet d'ajouter le dé du sort au jet d'une autre personne. Permet aussi de retrancher le dé du sort au jet d'un attaquant. Ce jet doit être fait avant l'action. & Chaque tour\\
Bouclier magique & Protège une seule personne au choix du magicien (incluant lui-même s'il le souhaite). Absorbe les dommages du dé de sort puis disparaît. & 1 fois par combat\\
\end{longtable}

\subsection{Furtivité et autres compétences de voleur}
\label{sec:orgcc8e35a}

Pour être furtif ou exercer d'autres talents du voleur, jetez votre dé de \textbf{Voleur} et consultez la table ci-dessous :

\begin{longtable}{l|c}
Compétences de voleur & Seuil à dépasser\\
\hline
\endfirsthead
\multicolumn{2}{l}{Suite de la page précédente} \\
\hline

Compétences de voleur & Seuil à dépasser \\

\hline
\endhead
\hline\multicolumn{2}{r}{Suite page suivante} \\
\endfoot
\endlastfoot
\hline
Se déplacer silencieusement & 2\\
Crocheter une serrure & 4\\
Escalader un mur & 4\\
Désamorcer un piège & 6\\
Sauter au dessus d'une fosse & 6\\
\end{longtable}

\subsection{Influence de l'équipement}
\label{sec:org9f8363c}
\subsubsection{Bonus dus à certains équipements}
\label{sec:org3efbb6d}

Certains matériels spécifiques vous donnent un bonus de "+D" (passage au dé supérieur) pour faire des choses spécifiques. Ce bonus vous donne droit à lancer le dé supérieur pour cette action spécifique.

\begin{longtable}{c|c|c}
Bonus & Dé de départ & Dé à utiliser\\
\hline
\endfirsthead
\multicolumn{3}{l}{Suite de la page précédente} \\
\hline

Bonus & Dé de départ & Dé à utiliser \\

\hline
\endhead
\hline\multicolumn{3}{r}{Suite page suivante} \\
\endfoot
\endlastfoot
\hline
+D & D4 & D6\\
+D & D6 & D8\\
+D & D8 & D10\\
+D & D10 & D12\\
\end{longtable}

\emph{Par exemple, si vous utilisez des bottes elfiques, vous pouvez passer de \textbf{Voleur} D4 à \textbf{Voleur} D6 quand vous furetez pour chercher des choses.}

\subsubsection{Autres équipements}
\label{sec:org9f5c86a}

Les cordes, crochets pour serrure, pelles, etc., peuvent être achetés, mais ils n'ont pas de dé assigné. Ainsi, il n'y a pas de limite quant aux objets de ce genre que les personnages peuvent transporter.

\section{Trésors et expérience}
\label{sec:org003c843}
\subsection{Trésors}
\label{sec:org16760dd}

Au cours des aventures, il est possible de trouver des trésors, par exemple :

\begin{longtable}{l|l}
Trésor & Caractéristique\\
\hline
\endfirsthead
\multicolumn{2}{l}{Suite de la page précédente} \\
\hline

Trésor & Caractéristique \\

\hline
\endhead
\hline\multicolumn{2}{r}{Suite page suivante} \\
\endfoot
\endlastfoot
\hline
Epée normale & Dommages : 1D6\\
Epée magique & Dommages : 1D8\\
Baguette magique & +D pour lancer les sorts\\
Bottes elfiques & +D pour se déplacer en silence\\
Potion & Contenant un sort ne fonctionnant qu'une seule fois\\
Pièces d'or (PO) & Permettent d'acheter des choses ou de gagner en expérience\\
\end{longtable}

\subsection{Expérience (avec règle optionnelle)}
\label{sec:orgd443b6d}

Les pièces d'or (PO) peuvent servir à acheter des chose, mais aussi à l'avancement du personnage.

\uline{Règle optionnelle} : Tuer des monstres rapporte des points d'expérience (notés "Exp" dans les tableaux de monstres). Ils définissent un compte de "Points d'Expérience", notés PE, et déterminés en fin d'aventure avec le maître de jeu. Ces points peuvent être consommés pour faire avancer le personnage (voir table ci\(_{\text{dessous}}\)).

\begin{longtable}{c|c|c}
Expérience & Gain & Règle originale\\
\hline
\endfirsthead
\multicolumn{3}{l}{Suite de la page précédente} \\
\hline

Expérience & Gain & Règle originale \\

\hline
\endhead
\hline\multicolumn{3}{r}{Suite page suivante} \\
\endfoot
\endlastfoot
\hline
100 PO & +D pour un dé (max : D12) & Oui\\
20 PO & +1 PV & Oui\\
500 PE & +D pour un dé (max : D12) & Non\\
100 PE & +1 PV & Non\\
\end{longtable}

\section{Monstres}
\label{sec:org4bbfbb7}

Principe : tous les monstres ont besoin d'un 4 ou plus pour toucher les personnages.

\subsection{Vermines}
\label{sec:org3134737}

Les personnages en touchent automatiquement un par attaque mais les vermines attaquent en groupe.

\begin{longtable}{p{3cm}|c|p{6cm}|c|p{1.3cm}|c|c}
Monstre & Attaque & Dommages & PV & Armure & Exp & Initiative\\
\hline
\endfirsthead
\multicolumn{7}{l}{Suite de la page précédente} \\
\hline

Monstre & Attaque & Dommages & PV & Armure & Exp & Initiative \\

\hline
\endhead
\hline\multicolumn{7}{r}{Suite page suivante} \\
\endfoot
\endlastfoot
\hline
Rat & D4 & Morsure 1 PV & 1 & - & 1 & +1\\
\hline
Araignée & D4 & Piqûre 1 PV & 1 & - & 1 & 0\\
\hline
Chauve souris vampire géante & D4 & Morsure 2 PV & 2 & - & 2 & +1\\
\hline
Moisissure gluante et puante & D4 & Erode le métal, détruit les armures et les épées & 25 & - & 25 & -\\
\hline
Eponge moisie magique & D4 & Les points de magie utilisés contre elle accroissent ses points de vie d'autant & 25 & - & 25 & -1\\
\end{longtable}

\newpage

\subsection{Monstres faibles}
\label{sec:org90bb888}

Les monstres faibles voyagent en bandes. Les personnages ont desoin d'un 2 ou plus pour les toucher.

\begin{longtable}{p{3cm}|c|p{6cm}|c|p{1.3cm}|c|c}
Monstre & Attaque & Dommages & PV & Armure & Exp & Initiative\\
\hline
\endfirsthead
\multicolumn{7}{l}{Suite de la page précédente} \\
\hline

Monstre & Attaque & Dommages & PV & Armure & Exp & Initiative \\

\hline
\endhead
\hline\multicolumn{7}{r}{Suite page suivante} \\
\endfoot
\endlastfoot
\hline
Rat géant & D6 & Morsure D4 & 4 & - & 4 & +1\\
\hline
Loup & D6 & Morsure D6 & 6 & - & 6 & 0\\
\hline
Goblin & D6 & Hache D8 & 8 & - & 8 & 0\\
\hline
Bandit de grand chemin & D6 & Epée D6 & 8 & 1d4 & 8 & 0\\
\end{longtable}

\subsection{Monstres moyens}
\label{sec:orgec213e1}

Les personnages ont desoin d'un 4 ou plus pour les toucher.

\begin{longtable}{p{3cm}|c|p{6cm}|c|p{1.3cm}|c|c}
Monstre & Attaque & Dommages & PV & Armure & Exp & Initiative\\
\hline
\endfirsthead
\multicolumn{7}{l}{Suite de la page précédente} \\
\hline

Monstre & Attaque & Dommages & PV & Armure & Exp & Initiative \\

\hline
\endhead
\hline\multicolumn{7}{r}{Suite page suivante} \\
\endfoot
\endlastfoot
\hline
Orc & D8 & Epée D6 & 10 & Bouclier D6 & 10 & 0\\
\hline
Soldat & D8 & Epée D6 & 10 & 1d8 & 10 & +1\\
\hline
Squelette guerrier & D8 & Hache D8 & 4 & - & 4 & -1\\
\hline
Araignée géante & D8 & Poison D4 par tour pendant 4 tours & 12 & - & 12 & -1\\
\hline
Zombie & D8 & Morsure D6 (guérie par le sort "soins") & 8 & - & 8 & 0\\
\hline
Homme poisson & D8 & Griffes D6, morsure D4 & 10 & Peau : 2 points & 10 & 0\\
\end{longtable}

\subsection{Monstres forts}
\label{sec:orgb469088}

Les personnages ont desoin d'un 6 ou plus pour les toucher.

\begin{longtable}{p{3cm}|c|p{3.5cm}|c|p{3.8cm}|c|c}
Monstre & Attaque & Dommages & PV & Armure & Exp & Initiative\\
\hline
\endfirsthead
\multicolumn{7}{l}{Suite de la page précédente} \\
\hline

Monstre & Attaque & Dommages & PV & Armure & Exp & Initiative \\

\hline
\endhead
\hline\multicolumn{7}{r}{Suite page suivante} \\
\endfoot
\endlastfoot
\hline
Géant & D10 & Gourdin D10 & 20 & - & 20 & -1\\
\hline
Troll & D10 & Mains D10 & 12 & Armure naturelle D10 & 12 & -1\\
\hline
Petit dragon & D10 & Pinces d6, morsure D8, souffle de feu D12 & 40 & Armure naturelle D6 & 40 & 0\\
\end{longtable}

\subsection{Sauve qui peut !}
\label{sec:orgfc975d0}

Les personnages ont desoin d'un 8 ou plus pour les toucher.

\begin{longtable}{l|c|p{3.5cm}|c|p{3.8cm}|c|c}
Monstre & Attaque & Dommages & PV & Armure & Exp & Initiative\\
\hline
\endfirsthead
\multicolumn{7}{l}{Suite de la page précédente} \\
\hline

Monstre & Attaque & Dommages & PV & Armure & Exp & Initiative \\

\hline
\endhead
\hline\multicolumn{7}{r}{Suite page suivante} \\
\endfoot
\endlastfoot
\hline
Grand dragon & D12 & Pinces D10, souffle de feu D12 & 60 & Armure naturelle D10 & 60 & -1\\
\end{longtable}

\newpage

\section{Equipement}
\label{sec:org0749790}

\subsection{Moins de 20 PO}
\label{sec:org7d075a3}

\begin{longtable}{l|c|p{10cm}}
Equipement & Prix (PO) & Commentaires\\
\hline
\endfirsthead
\multicolumn{3}{l}{Suite de la page précédente} \\
\hline

Equipement & Prix (PO) & Commentaires \\

\hline
\endhead
\hline\multicolumn{3}{r}{Suite page suivante} \\
\endfoot
\endlastfoot
\hline
Bougie & 1 & \\
Sac de couchage & 1 & \\
Gourde & 1 & \\
Sifflet & 1 & \\
Torche & 1 & \\
Sac à butin & 1 & \\
Repas pour une semaine & 5 & \\
Valise waterproof & 5 & \\
Corde de 3 mètres & 5 & \\
Briquet silex & 5 & \\
Sac à dos & 5 & \\
Pelle à creuser & 5 & +D pour creuser\\
Bandages & 5 & Soigne 1D4 une fois\\
Kit de l'aventurier & 10 & Sac à dos, briquet silex, sac de couchage, gourde, sac à butin\\
Lanterne & 10 & \\
Carte locale & 10 & \\
Corde & 10 & \\
Crochet escalade & 10 & +D en escalade si utilisé avec la corde\\
Marteau et piquets & 10 & \\
Parchemin, encre et plume & 10 & \\
Instrument de musique & 10 & \\
Baume soignant & 10 & Soigne 1D6 une fois\\
\end{longtable}

\subsection{20 PO et plus}
\label{sec:org525e8fe}

\begin{longtable}{l|c|p{10cm}}
Equipement & Prix (PO) & Commentaires\\
\hline
\endfirsthead
\multicolumn{3}{l}{Suite de la page précédente} \\
\hline

Equipement & Prix (PO) & Commentaires \\

\hline
\endhead
\hline\multicolumn{3}{r}{Suite page suivante} \\
\endfoot
\endlastfoot
\hline
Augmenter ses PV de 1 & 20 & \\
Tente pour 4 personnes & 20 & \\
Beaux habits & 20 & \\
Animal de compagnie & 20 & Chat, belette, hibou, faucon, etc.\\
Cape & 20 & +D pour se cacher\\
Gants pour escalader & 20 & +D escalade\\
Bottes elfiques & 20 & +D se déplacer en silence\\
Potion de soins & 20 & Soigne 1D12 une fois\\
Cheval harnaché & 50 & \\
Piège à ours & 50 & \\
Longue-vue & 50 & \\
Mirroir & 50 & \\
Crocket pour serrure & 50 & +D crochetage\\
Parchemin de sort & 50 & Contient un sort à usage unique\\
+D pour une caractéristique & 100 & Magicien, Guerrier ou Voleur\\
Chien de garde & 100 & Attaque D8. Dommages : morsure D6. 6 PV. Loyal.\\
Epée magique & 100 & +D pour le combat avec cette épée\\
Baguette magique & 100 & +D pour lancer des sorts\\
Etalon de guerrier & 100 & Attaque par piétinement D6. 12 PV. Féroce.\\
Laboratoire portable de magicien & 100 & Pour inventer de nouveaux sorts\\
\end{longtable}

\section{Feuille de personnage}
\label{sec:org64e2ce3}

\begin{itemize}
\item Version PDF : \href{https://github.com/orey/jdr/blob/master/DungeonSquad-fr/DungeonSquadFr-FeuillePerso.pdf}{Feuille de perso PDF}
\item Version ODP : \href{https://github.com/orey/jdr/blob/master/DungeonSquad-fr/DungeonSquadFr-FeuillePerso.odp}{Feuille de perso ODP}
\end{itemize}

\section{Ecran}
\label{sec:orgaa0d0d9}

\begin{itemize}
\item Version PDF : \href{https://github.com/orey/jdr/blob/master/DungeonSquad-fr/DungeonSquadFr-Ecran.pdf}{Ecran PDF}
\item Version ODP : \href{https://github.com/orey/jdr/blob/master/DungeonSquad-fr/DungeonSquadFr-Ecran.odp}{Ecran ODP}
\end{itemize}

\section{Générateur de rencontres}
\label{sec:org10ca7f2}

\subsection{Monstres niveau 1}
\label{sec:orgd6396f6}

Lancez 1D12.

\begin{longtable}{c|l|c}
Valeur & Monstre & Nombre\\
\hline
\endfirsthead
\multicolumn{3}{l}{Suite de la page précédente} \\
\hline

Valeur & Monstre & Nombre \\

\hline
\endhead
\hline\multicolumn{3}{r}{Suite page suivante} \\
\endfoot
\endlastfoot
\hline
1-2 & Rats & 1D8\\
3-4 & Chauves souris géantes & 1D8\\
5 & Araignée géante & 1D2\\
6 & Goblin & 1D2\\
7 & Rat géant & 1D4\\
8 & Squelette guerrier & 1D2\\
9 & Orc & 1D2\\
10 & Homme poisson & 1D2\\
11 & Eponge moisie & 1\\
12 & Zombie & 1D2\\
\end{longtable}

\subsection{Monstres niveau 2}
\label{sec:org1fa50b4}

Lancez 1D6.

\begin{longtable}{c|l}
Valeur & Monstre\\
\hline
\endfirsthead
\multicolumn{2}{l}{Suite de la page précédente} \\
\hline

Valeur & Monstre \\

\hline
\endhead
\hline\multicolumn{2}{r}{Suite page suivante} \\
\endfoot
\endlastfoot
\hline
1-2 & Troll\\
3-4 & Géant\\
5 & Petit dragon\\
6 & Relancer 1D6. 1-5 : Petit dragon, 6 grand dragon\\
\end{longtable}


\subsection{Trésors}
\label{sec:orgd5d7783}

Lancez 1D12.

\begin{longtable}{c|l}
Valeur & Trésor\\
\hline
\endfirsthead
\multicolumn{2}{l}{Suite de la page précédente} \\
\hline

Valeur & Trésor \\

\hline
\endhead
\hline\multicolumn{2}{r}{Suite page suivante} \\
\endfoot
\endlastfoot
\hline
1-2 & 1D6 PO\\
3-4 & 1D8 PO\\
5 & Des bijoux pour une valeur de 1D10 PO\\
6 & Potion de soins valable 1D4 fois\\
7 & Epée magique, dommages 1D8\\
8 & Baguette magique +D en sorts\\
9 & Parchemin bouclier magique (voir règles)\\
10 & Potion d'invisibilité, ne marche qu'une fois\\
11 & Kit de crochetage magique, marche 1D8 fois\\
12 & Torche éternelle, marche pour une aventure\\
\end{longtable}

\subsection{Difficultés et pièges}
\label{sec:orgbd61bb3}

Lancez 1D10.

\begin{longtable}{c|p{6cm}|p{3.5cm}|p{4cm}}
Valeur & Description & Seuil éviter / désamorcer & Dommages\\
\hline
\endfirsthead
\multicolumn{4}{l}{Suite de la page précédente} \\
\hline

Valeur & Description & Seuil éviter / désamorcer & Dommages \\

\hline
\endhead
\hline\multicolumn{4}{r}{Suite page suivante} \\
\endfoot
\endlastfoot
\hline
1 & Le sol tremble et la pièce s'effondre & Voleur 4 / Voleur 6 & 1D6\\
2 & Le plafond tremble et s'écroule & Voleur 4 / Voleur 6 & 1D6\\
3 & 1 flèche est tirée par un piège vers chaque personnage & Voleur 4 / Voleur 8 & 1D4\\
4 & Boule de feu venant d'un coin de la pièce & Voleur 4 / Voleur 6 & 1D6\\
5 & Trappe qui s'ouvre, besoin d'une corde pour sortir & Voleur 4 / Voleur 6 & 1D4\\
6 & Boule de pierre géante qui surgit d'un mur & Voleur 4 / Voleur 8 & 1D6\\
7 & Rayont paralysant & Voleur 4 / Voleur 6 & Paralysé pendant 1D6 heures - 1D4\\
8 & Cage emprisonnante & Voleur 4 / Voleur 6 & Prisonnier\\
9 & Moisissure gluante et puante & Voleur 3 / - & Erode le métal, les épées, les armures\\
10 & Porte fermée & - / Voleur 6 & \\
\end{longtable}

\subsection{Situations étranges}
\label{sec:org0f09948}

Lancez 1D10.

\begin{longtable}{c|l}
Valeur & Description\\
\hline
\endfirsthead
\multicolumn{2}{l}{Suite de la page précédente} \\
\hline

Valeur & Description \\

\hline
\endhead
\hline\multicolumn{2}{r}{Suite page suivante} \\
\endfoot
\endlastfoot
\hline
1 & Statue qui parle\\
2 & Nid de paille puant avec cafards\\
3 & Squelettes humains avec armes et épées\\
4 & Fontaine avec octogone\\
5 & Dortoir vide\\
6 & Chapelle du chaos avec autel\\
7 & Pièce avec grande cage et squellettes non humains\\
8 & Pièce avec vieux habits pourris\\
9 & Pièce avec un puits central\\
10 & Pièce avec une roue de pierre qui tourne\\
\end{longtable}

\vfill

\begin{center}
\includegraphics[width=3cm]{logo-orey-big.png}
\end{center}
\end{document}
