% Created 2022-01-01 Sat 21:34
% Intended LaTeX compiler: pdflatex
\documentclass[a4paper, 11pt, twoside]{article}
\usepackage[utf8]{inputenc}
\usepackage[T1]{fontenc}
\usepackage{graphicx}
\usepackage{grffile}
\usepackage{longtable}
\usepackage{wrapfig}
\usepackage{rotating}
\usepackage[normalem]{ulem}
\usepackage{amsmath}
\usepackage{textcomp}
\usepackage{amssymb}
\usepackage{capt-of}
\usepackage{hyperref}
\usepackage{baskervillef}
\usepackage{geometry}\geometry{ a4paper, total={170mm,257mm}, left=20mm, top=20mm,}
\usepackage{hyperref}\hypersetup{pdfauthor={Olivier Rey}, pdftitle={Ressources pour JDR}, pdfkeywords={jdr, ressources, orey-jdr}, pdfsubject={jeu de rôles}, pdfcreator={Emacs 26.1 (Org mode 9.1.9)}, pdflang={French}, colorlinks=true, linkcolor={blue}, urlcolor={blue}}
\usepackage{titlesec}\titlelabel{\thetitle. \quad}
\usepackage[table,svgnames]{xcolor}\rowcolors{1}{Gainsboro}{WhiteSmoke}
\usepackage{etoolbox}\AtBeginEnvironment{longtable}{\small}
\author{rey.olivier@gmail.com}
\date{2022-01-01}
\title{Ressources pour JDR}
\hypersetup{
 pdfauthor={rey.olivier@gmail.com},
 pdftitle={Ressources pour JDR},
 pdfkeywords={},
 pdfsubject={},
 pdfcreator={Emacs 26.1 (Org mode 9.1.9)}, 
 pdflang={Frenchb}}
\begin{document}

\maketitle
\tableofcontents

\begin{center}
\includegraphics[width=4cm]{logo-orey.png}
\end{center}

Les différents dossiers de ce repo contiennent des ressources en français pour JDR ou des traductions (voir aussi la \href{RessourcesPourJDR-ORey.pdf}{version PDF de cette page}).

Ce repo est le premier d'un cycle de 3. Voici les deux suivants :
\begin{itemize}
\item \href{https://github.com/orey/DandD}{github.com/orey/DandD} : Un repo en anglais dédié à D\&D, tout spécialement aux vieilles choses (0e et 1e),
\item \href{https://github.com/orey/ttrpg}{github.com/orey/ttrpg} : Un repo en anglais dédié à de nombreux jeux de rôles et contenant une grande liste de liens.
\end{itemize}

\section{Dernières livraisons}
\label{sec:org4fc682f}
\begin{longtable}{cp{2cm}p{1.5cm}p{4cm}ccp{4cm}}
\textbf{Date} & \textbf{JDR-RPG} & \textbf{Type} & \textbf{Livrable} & \textbf{Format} & \textbf{Ref} & \textbf{Commentaire}\\
2022 & \href{https://github.com/orey/jdr/tree/master/Mythic-fr}{Mythic} & GM Emulator & \href{https://github.com/orey/jdr/blob/master/Mythic-fr/MythicGME-EcranMJ-VersionFrancaise-OreyJdr05.pdf}{Ecran en français pour Mythic GME} & PDF & 05 & Traduction originale\\
2021 & \href{https://github.com/orey/jdr/tree/master/Fudge-fr}{Fudge} Department 13 & JDR & \href{https://github.com/orey/jdr/blob/master/Fudge-fr/Division13/Department13-FeuillePerso.pdf}{Feuille de perso Department 13} & PDF &  & Pour le setting Department 13\\
2021 & \href{https://github.com/orey/jdr/tree/master/Fudge-fr}{Fudge} & JDR & \href{https://github.com/orey/jdr/blob/master/Fudge-fr/FudgeEnUnePage/FudgeEnUnePage-OReyJdr04.pdf}{Fudge en une page} & PDF & 04 & Traduction originale\\
2021 & \href{https://github.com/orey/jdr/tree/master/FightingFantasys-fr}{Fighting Fantasy VF} & JDR & \href{https://github.com/orey/jdr/blob/master/FightingFantasys-fr/FightingFantasy-VersionFrancaise-OreyJdr03.pdf}{Fighting Fantasy Version Française} & PDF & 03 & Traduction et adaptation originale\\
2021 & Tous & Aide de jeu & \href{https://github.com/orey/jdr/blob/master/G\%C3\%A9n\%C3\%A9rateurLabyrinthe/GenerateurDeLabyrinthe-OreyJdr02.pdf}{Générateur de labyrinthe pour JDR} & PDF & 02 & Traduction et adaptation originale\\
2021 & \href{https://github.com/orey/jdr/tree/master/DungeonSquad-fr}{DungeonSquad! VF} & JDR pour enfants & \href{https://github.com/orey/jdr/blob/master/DungeonSquad-fr/DungeonSquad-VersionFrancaise-OreyJdr01.pdf}{DungeonSquad! Version Française} & PDF & 01 & Traduction et adaptation originale\\
2021 & \href{https://github.com/orey/jdr/tree/master/DungeonSquad-fr}{DungeonSquad! VF} & JDR pour enfants & \href{https://github.com/orey/jdr/blob/master/DungeonSquad-fr/DungeonSquadFr-FeuillePerso.pdf}{Feuille de perso} & PDF &  & Pour fille et garçon\\
2021 & \href{https://github.com/orey/jdr/tree/master/DungeonSquad-fr}{DungeonSquad! VF} & JDR pour enfants & \href{https://github.com/orey/jdr/blob/master/DungeonSquad-fr/DungeonSquadFr-Ecran.pdf}{Ecran} & PDF &  & Un outil indispensable\\
2021 & \href{https://github.com/orey/jdr/tree/master/AppelDeCthulhu}{Appel de Cthulhu} & Horreur & \href{https://github.com/orey/jdr/blob/master/AppelDeCthulhu/AppelDeCthulhu-EcranComplementaire.pdf}{Ecran complémentaire MJ} & PDF &  & Ecran complémentaire MJ\\
2021 & \href{https://github.com/orey/jdr/tree/master/AppelDeCthulhu}{Appel de Cthulhu} & Horreur & \href{https://github.com/orey/jdr/blob/master/AppelDeCthulhu/AppelDeCthulhu-FicheDeTemps.pdf}{Fiche de temps} & PDF &  & Pour l'Appel de Cthulhu ou autre jeu Basic RPS\\
2021 & Tous & Aide de jeu & \href{https://github.com/orey/jdr/blob/master/Aftermath/LocalisationDesBlessures.png}{Localisation des blessures} & PNG &  & A intégrer dans une synthèse d'aides de jeu pour MJ\\
2021 & \href{https://github.com/orey/jdr/tree/master/Risus-fr/}{Risus} & Flowchart & \href{https://github.com/orey/jdr/blob/master/Risus-fr/risus-flowchart.pdf}{Flowchart complet du jeu} & PDF &  & Peut servir d'éran\\
2021 & \href{https://github.com/orey/DandD}{D\&D 5e} & Couverture & \href{https://github.com/orey/DandD/blob/master/DandD\_5e\_BasicEditionLuluCover/Cover.pdf}{Couverture pour D\&D 5e Basic Rules} & PDF &  & Pour Lulu.com\\
\end{longtable}

\section{Liens}
\label{sec:org83c9c62}

\subsection{Sites de jeux en français}
\label{sec:orgc900177}

\begin{longtable}{p{7cm}p{7cm}}
\textbf{Type} & \textbf{Site}\\
\textbf{C} & \\
Le cénotaphe & \url{http://casquenoir.free.fr/index.php}\\
Créatures légendaires & \url{https://fr.wikipedia.org/wiki/Liste\_de\_cr\%C3\%A9atures\_l\%C3\%A9gendaires}\\
\textbf{D} & \\
Blog de Jérôme Darmont & \url{http://darmont.free.fr/}\\
Discussions de Rôlistes Ouvertes et Libres & \url{https://www.facebook.com/groups/254213402190606}\\
\textbf{E} & \\
Echecs: Check \& Strategy, site en français & \url{https://www.chess-and-strategy.com}\\
Empire Galactique JDR, un classique & \url{https://sites.google.com/site/empiregalact}\\
Egrégore, un JDR fantastique & \url{https://business.facebook.com/EgregoreJdR/?business\_id=456290144533916}\\
Epées et Sorcellerie JDR & \url{https://sites.google.com/site/wizardinabottle/epeesetsorcellerie}\\
\textbf{F} & \\
Une traduction française du RPG "FU" & \url{https://brunobord.gitbooks.io/fu-rpg-libre-et-universel/}\\
Faenix & \url{https://chezfaenyx.blogspot.com/2021/11/20-jeux-20-questions-episode-3.html}\\
Traduction française de Fudge & \url{http://fudge.ouvaton.org/}\\
\textbf{G} & \\
Giannirateur de scénarios & \url{http://loukoum.online.fr/jdr/adj/gianni1.htm}\\
 & \url{http://loukoum.online.fr/jdr/scenars/defi2012.htm\#47}\\
Le Grog, répertoire de JDR et news & \url{http://www.legrog.org/}\\
\textbf{H} & \\
Harry Potter JDR, un très beau travail & \url{https://www.geek-it.org/harry-potter-jdr}\\
Heroquest, un site de fan & \url{https://www.heroquest-revival.com}\\
\textbf{I} & \\
Imaginos & \url{https://blogs.bl0rg.net/imaginos/}\\
\textbf{K} & \\
Koma JDR et autres jeux de Xavier Raoult & \url{http://komajdr.free.fr/?page\_id=96}\\
\textbf{L} & \\
Les jeux d'Olivier Legrand & \url{http://storygame.free.fr/}\\
\textbf{M} & \\
Maléfices vieux suppléments & \url{https://www.scribd.com/user/381722775/Jean-Charles-BLANGENOIS}\\
\textbf{O} & \\
Osric JDR & \url{https://osric.fr}\\
\textbf{V} & \\
La voix d'Héort, ressources pour Glorantha & \url{https://heort.wordpress.com/}\\
 & \\
\end{longtable}

\subsection{Magazines en français}
\label{sec:org356f380}

\begin{longtable}{p{7cm}p{7cm}}
\textbf{Type} & \textbf{Site}\\
\textbf{B} & \\
Les anciens "Backstab" & \url{https://www.abandonware-magazines.org/affiche\_mag.php?mag=199}\\
\textbf{C} & \\
Les anciens "Casus Belli" & \url{https://www.abandonware-magazines.org/affiche\_mag.php?mag=188}\\
\textbf{G} & \\
Quelques vieux "Graal" & \url{https://www.abandonware-magazines.org/affiche\_mag.php?mag=402}\\
\textbf{J} & \\
Les anciens "Jeux et Stratégie", un must & \url{https://www.abandonware-magazines.org/affiche\_mag.php?mag=185}\\
\textbf{T} & \\
Les vieux "Tangente" & \url{https://www.abandonware-magazines.org/affiche\_mag.php?mag=326}\\
 & \\
\end{longtable}

\section{Explorations récentes}
\label{sec:org7668530}

A explorer : Fiasco, Nephilim, DCC.

\begin{longtable}{cp{2cm}p{1.5cm}p{7cm}ccc}
\textbf{Date} & \textbf{Game} & \textbf{Type} & \textbf{Comment} & \textbf{Note} & \textbf{OSR} & \textbf{Ongoing}\\
2021 & \href{https://github.com/orey/jdr/tree/master/BladesInTheDark-SRD}{Blades in the Dark SRD} & Heroic Fantasy &  & - & N & \textbf{Y}\\
2021 & \href{https://github.com/orey/jdr/tree/master/Risus-fr}{Risus} & Generic system & Irony with Clichés and D6 with difficulty factors & 3/5 & N & N\\
2021 & \href{https://www.drivethrurpg.com/product/89534/FU-The-Freeform-Universal-RPG-Classic-rules}{FU} & Generic system & Very basic system for roleplay & 3/5 & N & N\\
2021 & \href{http://www.onesevendesign.com/laserfeelings/}{Lasers and Feelings} & Sci-Fi & Great simple RPG & 4/5 & N & N\\
2021 & GURPS & Generic system & A great classical system with great supplements & 4/5 & N & \textbf{Y}\\
2021 & \href{https://github.com/orey/jdr/blob/master/Fudge-fr/FudgeEnUnePage-ORey03.pdf}{Fudge} (en une page) & Generic system & An "open GURPS" with a 7-levels ladder and scales. Very GURPS inspired & \textbf{5/5} & N & \textbf{Y}\\
2021 & \href{http://komajdr.free.fr/fichiers/BiTs.rar}{Bits } & Generic system & In French, a one page generic system & - & N & N\\
2021 & \href{http://storygame.free.fr/}{Trucs trop bizarres} & Modern kids & In French, a very simple game system & 3/5 & N & N\\
2021 & Advanced Fighting Fantasy & Heroic Fantasy & To play with children & - & N & Later\\
2021 & Modern AGE system & Modern & Ongoing & - & N & Later\\
2021 & Tunnels \& Trolls 1e & Heroic Fantasy & Interesting & 4/5 & N & N\\
2021 & Alternity 98 & Modern (Generic) & A very good system abandonned by WotC for the crappy D20 Modern & \textbf{5/5} & N & Later\\
2021 & The Esoterrorists 2e & Modern & The first Gumshoe system & - & N & Later\\
2021 & The Dragon & Press & Old issues of The Dragon, in \href{https://archive.org/details/DragonMagazine045\_201903}{archive.org} (1-100 251-280) & - & - & N\\
2021 & D20 Modern SRD & Generic System & Exploration in parallel to some \href{https://archive.org/details/Polyhedron105}{Polyhedron} readings & 2/5 & N & N\\
2021 & Gumshoe system SRD & Generic System & Entering into simplified translation process & - & N & Later\\
2021 & 13th Age & Heroic Fantasy & Just starting & - & Y & Later\\
2021 & Basic Roleplaying System & Generic System & The best, especially for CoC, free ed. is great & \textbf{5/5} & N & Later\\
2021 & The Wretched & Horror & Bof & 2/5 & N & N\\
2021 & GURPS & Generic System & Not convinced & 4/5 & N & N\\
2021 & Fighting Fantasy & Generic System & From Steve Jackson \& Ian Livingstone : \href{https://github.com/orey/jdr/tree/master/FightingFantasys-fr}{French translation} & 4/5 & Y & N\\
2021 & Bloodlust & Heroic Fantasy & French game by Croc & 3/5 & N & N\\
2021 & Metamorphosis Alpha & Sci-Fi & Interesting game & 3/5 & - & N\\
2021 & Ironsworn & Heroic Fantasy & Interesting game but too random (action dice vs 2D10) & 3/5 & N & N\\
2021 & Gumshoe system & Generic system & Investigation oriented: That one is for me :) & - & N & Later\\
2021 & DCC & Heroic Fantasy & A whole universe & 4/5 & Y & N\\
2021 & Légendes & Historic Fantasy & Great game for the universes. Hyper complex game system & 4/5 & N & Later\\
2021 & Tékumel & Heroic Fantasy & Author's world & 3/5 & N & N\\
2021 & Microlite & Generic System & \href{https://github.com/orey/jdr/tree/master/Microlite20-fr}{French translation} done. Not playable as-is. & 3/5 & N & N\\
2021 & Fortunes Wheel & - & Very interesting with tarot cards & - & N & Later\\
2021 & Maléfices & French Steampunk & Un des meilleurs JDR français & \textbf{5/5} & N & Later\\
2021 & GURPS & Generic System & To investigate & - & N & N\\
2021 & Traveller 1e & Sci-Fi & Seducing & - & N & Later\\
2020 & D\&D 5e basic rules & Heroic Fantasy &  & 3/5 & - & N\\
2020 & Covetous & GM Emulator & Bon produit avec plein de tables & - & N & Later\\
2020 & Conspiracy X & Modern &  & - & N & Later\\
2020 & D\&D SRD 3.5 & Heroic Fantasy & \href{https://github.com/orey/srd-3.5}{Repo spécial} avec diverses versions. & 4/5 & - & N\\
2020 & Méga & Sci-Fi & A French success & - & N & Later\\
2020 & Empire galactique & Sci-Fi & One of the first french RPG & 3/5 & N & N\\
2020 & L'appel de Cthulhu & Horror & The best & \textbf{5/5} & N & Later\\
2020 & Warhammer FR 1e & Heroic Fantasy & A very good game, surtout pour la Campagne Impériale & \textbf{5/5} & N & Later\\
2020 & Hero kids & RPG for kids & Bof, better play a simple adult game, or Bubblegumshoe & 2/5 & N & N\\
2020 & Pokethulhu & Fun & You need to like the comics & 2/5 & N & N\\
2020 & CRGE & GM Emulator & Based on the "Yes but\ldots{}/No but\ldots{}" & 2/5 & N & N\\
2020 & Mythic & GM Emulator & Great! \href{https://github.com/orey/jdr/tree/master/Mythic-fr}{Resources in French} (un écran !) & \textbf{5/5} & N & Later\\
2020 & PIP system & Generic system &  & - & N & Later\\
2020 & QAGS - Quick Ass Game System & Generic system & Simple and funny dynamic system & 4/5 & N & Later\\
2020 & Gateway & Heroic fantasy & Based on D\&D & 2/5 & Y & N\\
2020 & FU - Freeform Universal & Generic system & JDR basé sur le "Yes but\ldots{}/No but\ldots{}" & 3/5 & N & N\\
2020 & \href{https://github.com/orey/jdr/tree/master/Risus-fr}{Risus} & Generic system & In French:  \href{https://github.com/orey/jdr/tree/master/Risus-fr}{Règles résumées Risus} avec flowchart & 3/5 & N & N\\
2020 & PremièreFable (FirstFable) & JDR pour enfants & Traduction de FirstFable. Lien : \href{https://orey.github.io/premierefable/}{PremièreFable le JDR}. & 4/5 & N & N\\
2020 & \href{https://www.drivethrurpg.com/product/144558/Mini-Six-Bare-Bones-Edition}{MiniSix} & Generic system & D6 & - & N & Later\\
2020 & Dagger & RPG for kids & Bof & 2/5 & Y & N\\
\end{longtable}

\section{Systèmes de jeux}
\label{sec:orgc60243c}

Ci-dessous, quelques réflexions en vrac, notamment sur des marronniers du JDR.

\subsection{A propos des JDR modernes}
\label{sec:orga027caf}

Certains jeux récents, notamment la vague PbtA (\href{https://en.wikipedia.org/wiki/Powered\_by\_the\_Apocalypse}{Powered by the Apocalypse}), reprennent à leur compte des questions vieilles comme le JDR (du roleplay ou des règles, disait-on dans le temps) pour leur apporter des solutions posant question, en tous cas, à un vieux briscard comme moi.

Bien sûr, il faut relativiser. L'offre de TTRPG est si énorme aujourd'hui que tout ceci n'est peut-être qu'un n-ième épiphénomène destiné à faire vendre du papier\ldots{} pardon du PDF.

\subsubsection{Caractéristiques critiquées sur les JDR "anciens"}
\label{sec:org6c599f0}

Voici quelques uns des arguments que l'on trouve sur les forums concernant le pourquoi de la vague PbtA :
\begin{itemize}
\item Règles trop complexes, trop simulationnistes, trop spécifiques (une règle différente par cas sans unité globale), trop incohérentes (pas de ligne directrice globale), trop "crunchy" comme disent les américains. Les règles des anciens jeux font trop appel à des lancers de dés incessants, et à des modificateurs qui s'empilent de manière complexe, à des centaines de pages de règles.
\item Le MJ est trop directif et il ne met pas en place un univers collaboratif où les joueurs peuvent co-construire l'univers avec lui dans un mode collaboratif.
\item Le temps de préparation est trop long, trop important, pour les anciens jeux. L'investissement du MJ est trop important. L'investissement demandé n'est plus adapté à notre monde moderne, que ce soit pour le MJ ou pour les joueurs.
\item Certaines variantes de jeu ont permis de voir les choses différemment (JDR solo avec "gamemaster emulator", JDR sans MJ, etc.) et de pousser les jeux "narratifs" sur le devant de la scène.
\end{itemize}

C'est drôle, parce que ces arguments étaient les mêmes durant les années 80/90. Apocalypse World, le premier jeu PbtA date de 2010 soit plus de 20 ans après ces débats. La famille Baker, qui a designé ce jeu a la cinquantaine (en 2021), ils ont donc tenté de résoudre des choses qui ne leur plaisait pas.

Investiguons\ldots{} 

\subsubsection{Des PJ structurés avec des caractéristiques, des compétences et des nombres}
\label{sec:org3375c49}

La première composante du JDR est la composante PJ. Selon comment ces derniers sont structurés, les joueurs auront plus ou moins de possibilités. Le reste des règles de jeu organise les interactions entre les PJ et le monde, ainsi qu'avec les PNJ.

Les jeux de rôles de la première génération (D\&D, Cthulhu, GURPS, Rolemaster, etc.) étaient basés sur une possibilité de comparaison objective entre les PJ, et d'une manière de construire des PNJ permettant de dimensionner un adversaire (voir mes \href{https://github.com/orey/DandD}{commentaires sur D\&D}).

\subsubsection{Des PJ plus ou moins structurés avec des mots ou des expressions}
\label{sec:orge2a532e}

Avec l'apparition de jeux comme \href{https://github.com/orey/jdr/tree/master/Fudge-fr}{Fudge} (1992), un "pont" est dressé entre des valeurs sous forme de nombre (de -3 à +3) et des descriptifs qualificatifs portant cette "valeur". Même si la mécanique sous-jacente est encore à base de nombres et de modificateurs, les mots vont prendre progressivement une importance énorme dans le monde des JDR, jusqu'à prendre la place d'attributs, de compétences, de dons, de défauts ou même de pouvoirs.

\uline{Note} : Parler de Hurlements (1989).

Les mots s'imposent dans les créations des PJ au travers des "aspects", des "clichés", des "archétypes", des "avatars", des "concepts de personnages", etc. Ces mots peuvent être invoqués durant le jeu pour appeler une mécanique particulière, le plus souvent un bonus ou une compétence "sous-entendue" dans l'expression elle-même.

Or, au travers de cette irruption des mots dans les mécaniques des JDR, les problèmes relatifs à l'ambiguïté des mots et des expressions entrent dans le monde du JDR :

\begin{itemize}
\item Il faut une certaine expérience du JDR pour pouvoir définir des mots utiles au jeu ; en un sens, le JDR s'adresse implicitement à des vétérans, voire à des vétérans dans un mode jeu ironique (voire cynique) ; ou à des débutants qui n'ont pas de point de comparaison. Cela générera le besoin de "playbook", genre de nouvelle "classe de personnages".
\item Les mots ou phrases introduisent une incertitude autour des personnages, incertitude à laquelle le MJ doit s'adapter. En effet, les mots sont vagues, soumis à des interprétations et parfois en recouvrement sémantique partiel, ce qui rend compliqué leur usage. Cette ambiguïté est vue comme positive par les tenants de cette mécanique.
\item Les mots et les phrases introduisent aussi une incertitude dans l'équilibre des PJ entre eux, ainsi qu'avec les PNJ, et donc une possibilité d'arbitraire pour les MJ. Comment quantifier les expressions pour équilibrer les personnages ?
\item Les mots et les phrases favorisent les joueurs extravertis qui pourront interpréter de manière libre un "cliché" alors que les introvertis seront desservis par des règles basées sur des mots et sur le besoin d'improviser oralement pour les interpréter.
\item Ils créent des jeux très spécifiques dans lesquels le MJ doit investir pour comprendre comment utiliser ces mots ou expressions plus ou moins libres et plus ou moins contraintes par le game designer ; le plus souvent, il existe une mécanique sous-jacente à base de dés, mais moins présente que dans les anciens JDR.
\end{itemize}

\subsubsection{La structuration du "backgound" des personnages}
\label{sec:org72d09f5}

Le JDR narratif apporte une obsession étrange : celle de la \emph{structuration} du background. Dans la plupart des jeux anciens, même s'il était recommandé de créer un background à son personnage, cela était plus ou moins fait, et disons que le background s'enrichissait au fur et à mesure des parties, par exemple lors des démarrages d'aventures.

Beaucoup de jeux plus récents établissent une vraie dictature du background en exigeant de le structurer de manière schématique (voire caricaturale) en termes de jeu. Ainsi, on voit apparaître, en plus de la notion de "cliché" ou d'"archétype"  :
\begin{itemize}
\item Une certaine obsession pour les défauts des personnages, souvent utiles pour gagner des points dans le processus de création, parfois obligatoires dans le processus, souvent en contrepartie des dons ;
\item Des contacts sociaux obligatoires,
\item Un ennemi juré obligatoire,
\item Etc.
\end{itemize}

Cette schématisation à outrance du background des personnages concourt à en faire des caricatures, semblant être issues du même moule, et à rendre suffocant l'univers des personnages. Dans une conception ancienne, le PJ doit être libre avant tout et n'a pas à être "backgroundisé" arbitrairement au travers de dimensions caricaturales.

Ce sujet est délicat, car il s'agit d'un problème de curseur. La structuration du background des personnages a toujours été un sujet dans le JDR, mais dans des jeux comme PbtA, il semble que le bouchon soit poussé plus loin.

\subsubsection{Vers des caricatures de PJ ?}
\label{sec:org9852821}

Cette omniprésence des mots et des expressions dans les PJ a un effet pervers : elle construit des JDR où les personnages ont tendance à être des caricatures. Cette culture semble clairement influencée par :
\begin{itemize}
\item Les films et séries,
\item Les jeux vidéos,
\item Les comics et les mangas.
\end{itemize}

Or, jouer des caricatures ou des archétypes de personnages de films ou de BD est très limité et, pour moi, assez loin de la notion de jeu de rôle car plus prêt de la notion de jeu de plateau. Il y a un côté ennuyeux à cette approche. En fait, c'est comme si les jeux de super-héros avaient gagné en esprit sur le reste du JDR : il semble qu'il faille jouer des caricatures standardisées dans les jeux modernes, des personnages de série télé, qui ont été volontairement fabriqués pour évoluer dans un certain genre de séries très étroit (sous-genre).

\subsubsection{Les "move"}
\label{sec:orgfaacee8}

Dans certains jeux relativement récents, les mots sont utilisés pour les "move", des genres d'actions génériques que les PJ peuvent faire, ces actions étant adaptées au sous-genre proposé par le jeu. Le game designer impose, en plus de tout ce que nous avons vu, une certaine manière de jouer en déterminant les "move" par type de personnage ou pour tous les types. Généralement, ces moves sont spécifiques au jeu, peu faciles à comprendre et à utiliser, et en recouvrement sémantique les uns avec les autres (ce qui est volontaire dans certains jeux).

\subsubsection{Théâtre et jeu de plateau}
\label{sec:orgcd8ac0b}

Cela a deux conséquences :
\begin{enumerate}
\item Les personnages sont des archétypes (avec un playbook) aux actions archétypales (move).
\begin{itemize}
\item Cette vision va au delà des classes de personnages de D\&D pour qui la classe est un moyen de progresser dans une certaine voie.
\item Les aventures sont étudiées pour ces archétypes ayant ces actions archétypales, et il faut que le joueur endosse l'archétype et ses actions et le joue. S'il ne le joue pas, l'histoire peut être perturbée.
\item Nous sommes donc dans une perspective plus théâtrale que libre. Le canevas imposé aux joueurs semble plus dur que le canevas imposé par les anciens jeux.
\end{itemize}
\item Si les move devenaient des cartes physiques et les playbooks des fiches cartonnées, nous pourrions être dans un jeu de plateau, type jeu de stratégie.
\end{enumerate}

\subsubsection{Retour au jeu de stratégie ?}
\label{sec:org41f0105}

Si cette analyse est vraie, il est amusant de voir que nous vision donc, au sens strict du terme, une "régression" : là où Gary Gygax avait sorti le jeu de rôles du monde du jeu de plateau de stratégie, PbtA nous ramène vers le jeu de plateau avec les mêmes buts :
\begin{itemize}
\item Créer un jeu optimisé pour un sous-genre,
\item Créer des archétypes de personnages taillés pour le sous-genre et aux actions limitées pour obéir aux règles du sous-genre (dans l'esprit d'une "série"),
\item Contraindre les joueurs et le MJ à évoluer dans un narratif qui contraint les PJ dans une sandbox, dans un univers optimisé.
\end{itemize}

\subsubsection{Règles simplifiées au travers des move}
\label{sec:org4e0d4d8}

Comme tous les jeux de plateau, il est nécessaire de n'avoir qu'un ensemble de move restreints (ils portent d'ailleurs bien leur nom) pour évoluer dans l'univers. Cela a pour conséquence de faire des règles parfois brutales ou expéditives (et de manière surprenante) là où les anciens JDR proposaient des approches plus progressives (difficultés pour les joueurs à mesurer les risques d'une action, mécanique d'aggravation des situations suite à échec, etc.).

En fait, ce genre de jeux est tout à fait pertinent et il est normal qu'il séduise un certain public. Mais il est aussi normal que d'autres joueurs ne l'apprécient pas, car ce n'est plus tout à fait le même \emph{type} de jeu.

\subsubsection{Sommes-nous vraiment tous des game designers ?}
\label{sec:orge330b76}

Avec l'arrivée de plate-formes comme \href{https://www.kickstarter.com}{Kickstarter} ou \href{https://itch.io}{itch}, beaucoup de game designers se sont révélés, offrant une énorme diversité au JDR. Pour autant, la multiplication de l'offre fait apparaître des jeux \emph{dispensables} qui font se poser la question : est-ce qu'il y a autant de bons game designers sur le marché ?

Je n'en suis pas certain. Pour PbtA par exemple, les règles me semblent affreusement complexes et touffues, pleines de "trous dans la raquette" et nécessitant un investissement important pour tous les anciens MJ. Et pour quoi ? Pour faire du JDR narratif ? Mais on peut en faire facilement avec BRS ou même avec D\&D, dans des univers où les PJ sont moins caricaturaux et ont plus de possibilités de faire des choses et de s'adapter à la situation.

Pour ce qui est des scénarios ouverts, il faut se souvenir que bon nombre de scénarios anciens étaient très fouillés et très ouverts. Ils décrivaient l'univers de jeu, les PNJ, leurs motivations, le timing des événements et les PJ devaient s'insérer (voire bousculer) ces événements. Pour faire cela, il faut un système ouvert qui laisse la place belle aux inventions des joueurs, inventions qui ne manquaient pas d'arriver, souvent à la surprise du MJ.

\subsubsection{La manne des JDR PDF à l'heure de l'impression à la demande}
\label{sec:org9dcc49a}

Est-ce que le fait de lancer des systèmes complexes comme PbtA et de pouvoir lever de l'argent facilement ne rend pas plus facile l'arrivée sur le marché de produits immatures, pris dans l'engrenage financier de l'industrie du JDR ?

De plus, est-ce que les réseaux sociaux ne permettent pas de lancer des jeux dont les coûts de production sont très faibles et les revenus potentiellement très importants ? Vendre des PDF plus quelques impressions à la demande est une facilité qui collabore à mettre sur le marché des jeux parfois inaboutis.

Même chez les plus grands, le phénomène est réel. Comparons les suppléments GURPS 3e avec ceux de la 4e et vous verrez. Le digital a diminué la qualité, globalement, en poussant à la présence de suppléments, à l'exploitation du filon des suppléments sur une période courte, pendant la phase durant laquelle un jeu est à la mode.

\subsubsection{Le règne de D\&D 5e\ldots{} et de l'OSR}
\label{sec:org9e4c5d9}

Alors, oui, dans tout cette offre pléthorique, D\&D 5e règne en maître et, semble-t-il, dans la durée. D\&D, c'est un peu l'anti-jeu moderne. Même si son système de jeu a gagné en cohérence et a pris certains éléments des nouvelles tendances, D\&D est encore D\&D.

La mode OSR (Old-School Revival) pourrait être vu comme un genre de réaction à toutes ces innovations. J'ai lu dans des forums que les OSR-guys cherchaient une façon de jouer moins complexe, mais je ne suis pas d'accord. Ils veulent du "crunchy" de la grande époque, le sommet que nous aurons du mal à dépasser : AD\&D 1e ! Des règles énormes et pleines de cas particuliers, des tables à tiroirs comme Gary Gygax les aimait, des tas et des tas d'informations de toutes sortes un peu en vrac, une certaine inventivité pour les PJ, les monstres et les pièges, un sommet du genre.

\subsection{Système de jeu idéal}
\label{sec:org954ffbd}

Un équilibre entre :
\begin{itemize}
\item Possibilités de faire des jets de dés sous contraintes,
\item Simplicité et logique globale du système,
\item Adaptation à l'univers.
\end{itemize}

Par exemple, pour les charactéristiques, il est important qu'elles soient intuitives pour le MJ. Là dessus, D\&D et BRS sont au dessus du lot.

\subsection{Numéro 1 - Basic Role Playing system - BRP}
\label{sec:org5f9937e}

Le système Basic RP (\href{https://www.chaosium.com/brp-system-reference-document/}{SRD ici}), ou BRP, est un système très adaptable, logique et sans déformation de probabilités (contrairement au \href{https://github.com/orey/jdr/tree/master/D6-System}{système D6}). Il est particulièrement bien adapté aux univers fantasy, historiques et contemporains. Je ne connais pas d'implémentation du BRP en monde SF, en tous cas pas chez Chaosium.

\subsection{Numéro 2 - Maléfices}
\label{sec:orgd10a0cd}

\begin{itemize}
\item Un système de jeu Steampunk très adapté à l'univers.
\item Tarot très utile dans le jeu.
\item Un système un peu oublié.
\end{itemize}

\subsection{Numéro 3 - Donjons et Dragons}
\label{sec:orge46dadc}

D\&D possède un bon système de jeu qui a fait ses preuves dans une multitude de versions. Son système est simple et basé sur le paradigme suivant : \texttt{D20 + modificateurs >= Classe de difficulté} (par exemple, dépendant plus ou moins directement de la classe d'armure). Ce système a l'avantage de ne pas tordre les probabilités (contrairement au \href{https://github.com/orey/jdr/tree/master/D6-System}{système D6}).

Voir \href{https://github.com/orey/jdr/tree/master/DandD}{la page dédiée}.

\subsection{Articles sur les probabilités dans le JDR}
\label{sec:org41fab51}

\begin{itemize}
\item Une analyse des problèmes de probabilités du système D6 : voir \href{https://github.com/orey/jdr/tree/master/D6-System}{le folder D6-system}
\item Une analyse des probabilités de l'étrange système de jeu de IronSworn : voir \href{https://github.com/orey/jdr/tree/master/IronSworn}{le folder IronSworn}
\end{itemize}

\subsection{Eléments de systèmes de jeu intéressants}
\label{sec:org9d3b757}
\subsubsection{Fudge}
\label{sec:orge30ff21}

Deux éléments sont vraiment originaux :
\begin{itemize}
\item Le premier est l'usage de mots pour décrire les niveaux des caractéristiques et des skills. Fudge est sans doute un des premiers jeux à avoir fait cela (même si en fait, la mécanique de jeux reste sous-jacente et basé sur des nombres).
\item Le second est la gestion des échelles, qui est une vraie originalité de Fudge. Il est, en effet, possible de faire lutter des PJ et PNJ appartenant à des échelles différents. Une innovation très intéressante car, sur ce point, la mécanique est bien aboutie.
\end{itemize}

Voir l'\href{https://github.com/orey/jdr/tree/master/Fudge-fr}{article détaillé}.

\subsubsection{Tunnels \& Trolls}
\label{sec:org805bb55}

Le combat de groupe (mêlée) est un vrai combat de groupe :
\begin{itemize}
\item Les attaques de tous les joueurs sont cumulées,
\item Les attaques de tous les monstres le sont aussi,
\item On fait la différence (contest) pour calculer les dégâts à répartir sur la partie concernée.
\end{itemize}

Malin et efficace.

\subsubsection{Bloodlust}
\label{sec:orgd3983d3}

\begin{enumerate}
\item Mécanisme de combat
\label{sec:org051ee09}

Une seule table pour attaquant vs défenseur. En abscisse et en ordonnée :
\begin{itemize}
\item Attaque brutale
\item Attaque normale
\item Attaque rapide
\item Parade
\item Esquive
\end{itemize}

Dans le combat, chacun est tour à tour attaquant et défenseur. Fluide et efficace.

En bref, le combat est comme un double "contest" avec des modificateurs. C'est assez malin.

\item Réussites et échecs critiques
\label{sec:org5fcbdf8}

Bloodlust est un système à pourcentage. En cas de réussite, si l'unité de la valeur du jet est 0, on est dans un cas de réussite critique. Pareil pour les échecs critiques avec une valeur de l'unité de 1 sur le jet de pourcentage raté.
\end{enumerate}
\end{document}
