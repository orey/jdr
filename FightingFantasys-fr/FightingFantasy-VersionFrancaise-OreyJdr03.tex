% Created 2022-01-01 Sat 02:11
% Intended LaTeX compiler: pdflatex
\documentclass[a4paper, 11pt, twoside]{article}
\usepackage[utf8]{inputenc}
\usepackage[T1]{fontenc}
\usepackage{graphicx}
\usepackage{grffile}
\usepackage{longtable}
\usepackage{wrapfig}
\usepackage{rotating}
\usepackage[normalem]{ulem}
\usepackage{amsmath}
\usepackage{textcomp}
\usepackage{amssymb}
\usepackage{capt-of}
\usepackage{hyperref}
\usepackage{baskervillef}
\usepackage{geometry}\geometry{ a4paper, total={170mm,257mm}, left=20mm, top=20mm,}
\usepackage{hyperref}\hypersetup{pdfauthor={Olivier Rey}, pdftitle={Dungeon Squad! - Version Française}, pdfkeywords={jdr, dungeonsquad}, pdfsubject={jeu de rôles}, pdfcreator={Emacs 26.1 (Org mode 9.1.9)}, pdflang={Frenchb}, colorlinks=true, linkcolor={blue}, urlcolor={blue}}
\usepackage[french, frenchb]{babel}
\usepackage{titlesec}\titlelabel{\thetitle. \quad}
\usepackage[table,svgnames]{xcolor}\rowcolors{1}{Gainsboro}{WhiteSmoke}
\usepackage{etoolbox}\AtBeginEnvironment{longtable}{\small}
\author{Olivier Rey}
\date{2021-10-09}
\title{Fighting Fantasy - Règles en français}
\hypersetup{
 pdfauthor={Olivier Rey},
 pdftitle={Fighting Fantasy - Règles en français},
 pdfkeywords={},
 pdfsubject={},
 pdfcreator={Emacs 26.1 (Org mode 9.1.9)}, 
 pdflang={Frenchb}}
\begin{document}

\maketitle
\tableofcontents

\newpage

\begin{center}
\includegraphics[width=4cm]{FF2018.png}
\end{center}

\section{Fighting Fantasy - Version Française}
\label{sec:orgb2036cd}

\subsection{Introduction}
\label{sec:org3b872d4}

Fighting Fantasy est une série de livres de jeux de rôles pour un seul joueur créée par Steve Jackson (le \href{https://en.wikipedia.org/wiki/Steve\_Jackson\_(British\_game\_designer)}{Steve Jackson} anglais et non \href{https://en.wikipedia.org/wiki/Steve\_Jackson\_(American\_game\_designer)}{celui de GURPS} qui est américain) et \href{https://en.wikipedia.org/wiki/Ian\_Livingstone}{Ian Livingstone}. Le premier volume de la série fut publié en livre de poche par Pufin en 1982.

Cette série propose un système de jeu de rôles très simple qui fut ensuite complété de diverses façons, notamment dans le livre \emph{Dungeoneer, Advanced Fighting Fantasy}. Récemment, le système a été repris et customisé dans le jeu de rôles \href{https://melsonian-arts-council.itch.io/troika-numinous-edition}{Troika!}.

Nous avons repris les principales règles de  \emph{Dungeoneer, Advanced Fighting Fantasy} pour proposer un jeu très simple, jouable avec tous les types de joueurs, noamment les joueurs débutants et les enfants.

\begin{longtable}{ll}
Concepteur & (C) Steve Jackson\\
Version originale & Version originale 1984 - Copyright Steve Jackson\\
Website & \href{https://www.fightingfantasy.com/}{https://www.fightingfantasy.com/}\\
Traduction et adaptation & (C) O. Rey 2021\\
Version & 1.1\\
Octobre 2021 & Corrections diverses\\
 & Mise en place des tags pour génération Latex/PDF\\
\end{longtable}

\subsection{Ce dont vous avez besoin pour jouer}
\label{sec:org795952a}

Chaque joueur doit avoir accès à :
\begin{itemize}
\item Un crayon et du papier,
\item Trois dés à 6 faces par joueur,
\item Des choses à picorer et des boissons.
\end{itemize}

\subsection{Notes}
\label{sec:orgf2eb6b9}

Vous pouvez utiliser ce jeu avec d'autres ressources :
\begin{itemize}
\item Un \href{https://github.com/orey/jdr/tree/master/G\%25C3\%25A9n\%25C3\%25A9rateurLabyrinthe}{générateur de labyrinthe} ;
\item Il est aussi possible de l'utiliser sans MJ, en jouant avec des émulateurs de MJ, tels \href{https://github.com/orey/jdr/tree/master/Mythic-fr}{Mythic}.
\end{itemize}

\section{Création du personnage}
\label{sec:org56ba3f2}

\subsection{Caractéristiques}
\label{sec:org313842b}

Un personnage possède trois caractéristiques : \textbf{Compétences} (Skill), \textbf{Points de Vie} (Stamina)  et \textbf{Chance} (Luck).

\uline{Note} : Dans les \emph{Livres Dont Vous Etes le Héros}, "Skill" est traduit par "Habileté" et "Stamina" par "Constitution". Nous préférons la traduction "Compétences" pour "Skill" car elle plus large que la seule notion d'habileté ; elle est au centre des combats, et couvre les autres compétences. Pour ce qui est de "Stamina", même si la traduction "Constitution" est fidèle, la notion de "Point de Vie" est plus idiomatique du jeu de rôle français.

\newpage

\begin{longtable}{lcc}
\textbf{Caractéristique} & \textbf{Création} & \textbf{Acronyme}\\
Compétences & 1D6+6 & COMP\\
Points de vie & 2D6+12 & PV\\
Chance & 1D6+6 & CHA\\
\end{longtable}

S'il tombe à 0 PV ou moins, le PJ meurt.

\subsection{Test de chance}
\label{sec:org518167e}

Pour faire un test de Chance, jetez 2D6.

\begin{longtable}{cl}
\textbf{Test de chance} & \textbf{Résultat}\\
2D6 <= CHA & Test de chance réussi, -1 CHA\\
2D6 > CHA & Test de chance échoué, -1 CHA\\
\end{longtable}

Après chaque test de Chance, enlevez 1 point à la Chance.

\subsection{Restaurer les caractéristiques}
\label{sec:org400c5c9}

\begin{longtable}{cll}
\textbf{Caractéristique} & \textbf{Moyen} & \textbf{Effet}\\
Compétences & Potion de Compétences & Retour à la valeur initiale\\
Points de Vie & Potion de Vie & Retour à la valeur initiale\\
Chance & Potion de Chance & Restaure à la valeur initiale + 1 point CHA\\
\end{longtable}

\section{Equipement}
\label{sec:org13f2276}

Tous les personnages démarrent avec les objets suivants :

\begin{longtable}{lcl}
\textbf{Objet} & \textbf{Points/Doses} & \textbf{Commentaires}\\
Une épée & - & -\\
Un sac à dos & - & Pour mettre les trésors\\
Une lanterne & - & Une lanterne et son combustible suffisent pour une aventure\\
Des provisions & 10 repas & Chaque repas (-1 repas) restaure 4 PV\\
 &  & On ne peut pas consommer les provisions pendant un combat\\
Une potion & 2 doses & Une potion de Compétences, de Vie ou de Chance (au choix du joueur)\\
 &  & Voir la section "restaurer les compétences"\\
\end{longtable}

\section{Monstres}
\label{sec:org676fd90}

\subsection{Attaques}
\label{sec:orge4f364e}

Les montres ont aussi ont des caractéristiques \textbf{Compétences} et \textbf{Points de Vie}, mais ils n'ont pas de caractéristique Chance.

Ils ont une autre caractéristique : \textbf{Attaques} (ATT) qui représente le nombre de joueurs que les monstres peuvent attaquer \emph{en même temps}.

\subsection{Exemple de monstre}
\label{sec:org961bbfe}

Le loup-garou est un monstre à deux ATT : il est donc capable d'attaquer au plus 2 PJ dans un même round de combat.

\newpage

\begin{longtable}{lccc}
 & \textbf{COMP} & \textbf{PV} & \textbf{ATT}\\
Loup-garou & 8 & 9 & 2\\
\end{longtable}

\section{Combats}
\label{sec:org15a2fbe}

\subsection{Modificateurs de Compétences}
\label{sec:orgfaff590}

Certains objets (comme une épée magique) peuvent apporter des modificateurs aux Compétences.

Certaines règles s'appliquent :
\begin{itemize}
\item On ne peut utiliser qu'une seule arme en combat (et donc potentiellement un seul bonus).
\item La valeur de Compétences ne peut jamais excéder sa valeur initiale.
\end{itemize}

\subsection{Combat simple}
\label{sec:org5dcd74a}

Un round de combat se passe comme suit :
\begin{enumerate}
\item Les joueurs et les monstres attaquent en même temps en calculant leur score en \textbf{Attaque} (ATT) ;
\begin{itemize}
\item ATT monstre = 2D6 + COMP
\item ATT PJ = 2D6 + COMP
\end{itemize}
\item Les attaques sont ensuite comparées entre elles.
\end{enumerate}

\begin{longtable}{lcc}
\textbf{Action} & \textbf{Effet} & \textbf{Option joueur : Jet de chance}\\
Si ATT monstre > ATT PJ & -2 PV pour le PJ & Voir table suivante cas 1\\
Si ATT PJ > ATT monstre & -2 PV pour le monstre & Voir table suivante cas 2\\
Si ATT PJ = ATT monstre & Aucun effet & -\\
\end{longtable}

Le joueur peut décider d'utiliser sa chance, soit pour éviter un coup donné par un monstre (cas d'un échec du PJ en combat), soit pour aggraver la blessure du monstre (cas du succès du PJ en combat).

\begin{longtable}{ccl}
\textbf{Cas} & \textbf{Jet de chance} & \textbf{Effet}\\
1 & Réussi & -1 PV au lieu de -2 PV pour le PJ, -1 CHA\\
1 & Raté & -3 PV au lieu de -2 PV pour le PJ, -1 CHA\\
2 & Réussi & -4 PV au lieu de -2 PV pour le monstre, -1 CHA\\
2 & Raté & -1 PV au lieu de -2 PV pour le monstre, -1 CHA\\
\end{longtable}

Le combat s'arrête quand l'un des deux adversaire est mort (PV = 0) ou s'enfuit.

Si un PJ s'enfuit, il perd automatiquement 2 PV (dernière blessure infligée par le monstre). La chance peut être utilisée pour réduire les dommages (voir table ci-dessus cas 1).

\subsection{Combat multiple}
\label{sec:org6eea728}

Le combat multiple est assez amusant et logique. Nous l'exposons au travers d'exemples.

\subsubsection{Cas d'un monstre possédant une seule ATT contre trois PJ (A, B et C)}
\label{sec:org8ac368a}

\begin{longtable}{cl}
\textbf{Séquence} & \textbf{Action}\\
1 & Le MJ tire au sort le PJ qui sera attaqué (ou le choisit), disons C\\
2 & Combat simple entre le PJ et le monstre\\
 & Le MJ note le score de ATT du monstre pour ce tour\\
3 & Les autres PJ peuvent attaquer le monstre (ici A et B)\\
 & Si ATT PJ > ATT monstre : -2 PV pour le monstre\\
 & Si ATT PJ <= ATT monstre : le monstre n'a rien\\
 & A et B ne peuvent pas prennent aucun dommage\\
\end{longtable}

On appelle les attaques de A et B, des \textbf{attaques protégées}, car ces derniers ne peuvent pas prendre de dommages.

Au round suivant, le processus recommence.

\subsubsection{Cas d'un monstre à 8 ATT contre quatre PJ (A, B, C et D)}
\label{sec:org661c1b5}

\uline{Note}: si le nombre d'attaques du monstre est supérieure au nombre de PJ, cela ne signifie pas que le monstre a des attaques supplémentaires. Le nombre d'attaques correspond au nombre maximum de PJ que le monstre peut attaquer. Dans le cas présent, le monstre ne pourra attaquer que les 4 PJ.

\begin{longtable}{cl}
\textbf{Séquence} & \textbf{Action}\\
1 & Le MJ calcule le score ATT du monstre (2D6 + COMP)\\
 & Ce nombre  est valable pour le round pour tous les combats avec tous les PJ\\
2 & Chaque combat est résolu normalement.\\
\end{longtable}

\subsubsection{Cas de deux PJ (A et B) contre deux monstres (X ayant 2 ATT et Y ayant 1 ATT)}
\label{sec:org566ae43}

\begin{longtable}{cl}
\textbf{Séquence} & \textbf{Action}\\
1 & Le MJ demande aux joueurs quels monstres ils veulent attaquer (ex: A-X et B-Y).\\
 & Les monstres répondront aux attaques des PJ.\\
 & Les combats doivent donc se dérouler entre A et X, et B et Y.\\
2 & Résoudre les combats A-X et B-Y.\\
3 & X a une seconde attaque, il peut donc attaquer B en mode attaque protégée.\\
\end{longtable}

Tout monstre supplémentaire attaquera de manière aléatoire l'un des deux PJ.

\section{Situations communes}
\label{sec:orgf715a53}
\subsection{Soudoyer/corrompre}
\label{sec:orgcca7971}

Les monstres un peu intelligents aiment l'or. Le MJ peut accepter que les PJ tentent de les corrompre. Le MJ décide d'une probabilité de réussite et lance 1D6 (1 sur 6, ou 3 sur 6, etc.). Les monstres peuvent donner quelques informations s'ils se font corrompre.

\subsection{Equipement des PJ}
\label{sec:orga3ad709}

Les PJ ne peuvent pas transporter un nombre illimité de choses. Un PJ ne devrait pas transporter plus de 10 articles d'équipement (hors or et provisions). Les gros objets comptent pour plus d'un point. Le MJ doit être vigilant sur ce point.

\subsection{Portes}
\label{sec:orga76cf28}

\begin{longtable}{ll}
\textbf{Type} & \textbf{Commentaire}\\
Porte magique & Ont besoin d'un sort pour être ouvertes (ou sous contrôle d'un sorcier)\\
Porte ordinaire & Jeter 1D6 : 1-2 la porte est fermée ; 3-6 la porte est ouverte\\
Casser une porte ordinaire & Jet réussi de 2D6 strictement sous COMP ; -1 PV\\
 & Si le jet est supérieur ou égal à COMP, la porte résiste ; -1 PV\\
 & Deuxième tentative : 2D6 + 1 strictement sous COMP pour réussir ; -1 PV\\
 & Troisième tentative : 2D6 + 2 strictement\ldots{} (etc.)\\
Portes secrètes & Le PJ doit chercher ; le MJ jette 2D6 sous la COMP du PJ\\
 & Si le jet est réussi, la porte est trouvée (mais pas ouverte)\\
 & Jet de CHA pour trouver comment l'ouvrir\\
\end{longtable}

\subsection{Fuite}
\label{sec:org0a42a6c}

Le MJ doit décider si la fuite est possible (par exemple PJ acculé). Si la fuite est possible, la règle ci-dessus (combat simple) s'applique. Idem pour les monstres (intelligents) qui fuient.

\newpage

\subsection{Chute}
\label{sec:orgc95fd69}

\begin{longtable}{ll}
\textbf{Hauteur} & \textbf{Commentaire}\\
Inférieur à 2m & Pas de dommages\\
Par tranche de 10m & Faire un jet 2D6 + 1 sous CHA\\
 & Ex : 10m, 2D6 + 1 sous CHA ; 30m, 2D6+3 sous CHA\\
 & Si jet de CHA raté, le PJ est blessé. Perte de PV : 1 + 1 par 10m\\
\end{longtable}

\subsection{Perte d'un arme}
\label{sec:orga44713e}

Si un PJ perd son arme, sa COMP est diminuée de 4 jusqu'à ce qu'il trouve une autre arme.

\subsection{Mouvement}
\label{sec:org47fb50e}

Laissé à l'arbitrage du MJ et suivant les situations (longs couloirs avec pièges).

\subsection{Ouvrir un coffre}
\label{sec:org13c8d4c}

Similaire aux portes :
\begin{itemize}
\item Un coffre a 5 chances sur 6 d'être fermé.
\item Pour ouvrir le coffre : 2D6 strictement sous COMP
\item Si le PJ retente, à chaque essai, son arme s'abîme et le PJ perd un point de COMP par tentative jusqu'à ce qu'il trouve une autre arme.
\end{itemize}

Pour trouver les compartiments secrets dans les coffres, le PJ doit chercher le compartiment. La règle des portes secrètes s'applique.

\subsection{Pickpocket}
\label{sec:orgf001333}

Un jet de COMP strictement réussi est un succès. Le MJ peut donner un malus (6-8 est un malus acceptable si la situation ne se prête pas à jouer au pickpocket).

\subsection{Provision}
\label{sec:orgc430b49}

Les PJ peuvent consommer leurs provisions à tout moment sauf dans un combat.

Le nombre de provisions dont bénéficient les PJ au départ de l'aventure dépend de différents facteurs : longueur de l'histoire, provisions disponibles dans le scénario, etc.

\begin{longtable}{cc}
\textbf{Aventure} & \textbf{Nb de provisions}\\
Courte & 2\\
Moyenne & 4\\
Longue & 6+\\
\end{longtable}

\subsection{Chercher}
\label{sec:org00a6814}

Le PJ doit dire ce qu'il cherche. Le MJ fait les jets de dés : 2D6 strictement sous COMP pour trouver.

\subsection{Se déplacer en silence}
\label{sec:org7d23456}

2D6 strictement sous COMP. Le MJ peut ajouter des malus.

\subsection{Monstres errants}
\label{sec:orgd90c98a}

Si les PJ s'attardent trop dans un lieu, il est possible de générer une rencontre avec un monstre errant. le MJ lance 1D6 régulièrement. Si c'est un 1, un monstre a repéré les PJ.

\newpage

En souterrain :

\begin{longtable}{clccc}
\textbf{1D6} & \textbf{Créature} & \textbf{COMP} & \textbf{PV} & \textbf{ATT}\\
1 & Goblin & 5 & 3 & 1\\
2 & Orc & 6 & 3 & 1\\
3 & Gremlin & 6 & 3 & 1\\
4 & Rat géant & 5 & 4 & 1\\
5 & Squelette & 6 & 5 & 1\\
6 & Troll & 8 & 7 & 3\\
\end{longtable}

En extérieur :

\begin{longtable}{clccc}
\textbf{1D6} & \textbf{Créature} & \textbf{COMP} & \textbf{PV} & \textbf{ATT}\\
1 & Goblin & 5 & 3 & 1\\
2 & Chauve-souris géante & 5 & 4 & 1\\
3 & Rat Géant & 5 & 4 & 1\\
4 & Chien de guerre & 7 & 6 & 1\\
5 & Loup-garou & 8 & 9 & 2\\
6 & Ogre & 8 & 10 & 2\\
\end{longtable}


\vfill

\begin{center}
\includegraphics[width=3cm]{logo-orey-big.png}
\end{center}
\end{document}
